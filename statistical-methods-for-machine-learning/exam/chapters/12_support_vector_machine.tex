\newpage
\section{Support Vector Machine}

\begin{itemize}
    
    \item Write the convex optimization problem with linear constraints that defines the SVM hyperplane in the linearly separable case.\\
        
        Given a linearly separable training set $S = \{(\boldsymbol{x}_1, y_1), \dots, (\boldsymbol{x}_n, y_n)\} \in \mathbb{R}^d \times \{-1, 1\}$, SVM outputs the linear classifier corresponding to the unique solution $\boldsymbol{w}^*$ of the following convex optimization problem with linear constraints:
            \begin{align*} 
                & \underset{\boldsymbol{w} \in \mathbb{R}^d}{\min} \ \frac{1}{2} \Vert \boldsymbol{w} \Vert^2 \\
                & \text{s.t.} \ y_t \boldsymbol{w}^\top \boldsymbol{x}_t \geq 1 \ \text{for} \ t = 1, \dots, m 
            \end{align*}

            Geometrically, $\boldsymbol{w}^*$ corresponds to the \textbf{maximum margin separating hyperplane}. For every linearly separable set $S = \{(\boldsymbol{x}_1, y_1), \dots, (\boldsymbol{x}_n, y_n)\} \in \mathbb{R}^d \times \{-1, 1\}$, the maximum margin is defined as 
            $$
            \gamma^* = \underset{\boldsymbol{u} : \Vert \boldsymbol{u} \Vert = 1}{\max} \underset{t = 1, \dots, m}{\min} \ y_t \boldsymbol{u}^\top \boldsymbol{x}_t
            $$ 

            and the vector $\boldsymbol{u}^*$ achieving the maximum maring is the maximum margin separator.

    \item Write the unconstrained optimization problem whose solution defines the SVM hyperplane when the training set is not necessarily linearly separable.\\

        If we consider the case of a non-linearly separable training set, we should analyze how the SVM objective changes. Consider the following formulataion
        \begin{align*}
            \underset{(\boldsymbol{w}, \boldsymbol{\xi}) \in \mathbb{R}^d+m}{\min} \quad & \frac{\lambda}{2} \Vert \boldsymbol{w} \Vert^2 + \frac{1}{m} \sum_{t = 1}^m \xi_t \\
            \text{s.t.} \quad & y_t \boldsymbol{w}^\top \boldsymbol{x}_t \geq 1 - \xi_t & t = 1, \dots, m \\
            & \xi_t \geq 0 \ \text{for} & t = 1, \dots, m
        \end{align*}

        The components of $\boldsymbol{\xi} = (\xi_1, \dots, \xi_m)$ are called \textbf{slack variables} and measure how each margin constraint is violated by a potential solution $\boldsymbol{w}$. Finally, a regularization parameter $\lambda > 0$ is introduced to balance the two terms.\\

        We now consider the constraints involving the slack variables $\xi_t$. In oreder to minimize each $\xi_t$ we can set
        
        \begin{equation}
            \xi_t =
            \begin{cases}
                1 - y_t \boldsymbol{w}^\top \boldsymbol{x}_t & \text{if} \ y_t \boldsymbol{w}^\top \boldsymbol{x}_t < 1 \\                
                0 & \text{otherwise}
            \end{cases}
        \end{equation}
        
        Now, fix $\boldsymbol{w} \in \mathbb{R}^d$, we can see $\xi_t = \left[1 - y_t \boldsymbol{w}^\top \boldsymbol{x}_t \right]_+$ which is the hinge loss  $h_{t}(\boldsymbol{w})$.\\

        The SVM problem can be rewritten as $$\underset{\boldsymbol{w} \in \mathbb{R}^d}{\min} \ \frac{\lambda}{2} \Vert \boldsymbol{w} \Vert^2 + \frac{1}{m} \sum_{t = 1}^m h_{t}(\boldsymbol{w})$$.\\


    \newpage
    \item Write the bound on the expected value of the SVM objective function achieved by Pegasos.
Provide also a bound on the expected squared norm of the loss gradient.\\
        
        $$
        \mathbb{E}\left[F(\bar{\boldsymbol{w}})\right] \leq F(\boldsymbol{w}^*) + \frac{\mathbb{E}\left[G^2\right]}{2 \lambda T} (\ln T + 1)
        $$
            
        where:
        \begin{itemize}
            \item $F(\boldsymbol{w}) = \frac{\lambda}{2} \Vert \boldsymbol{w} \Vert^2 + \frac{1}{m} \sum_{t = 1}^m h_{t}(\boldsymbol{w})$ is the SVM objective function, where $h_{t}(\boldsymbol{w}) = \left[1 - y_t \boldsymbol{w}^\top \boldsymbol{x}_t \right]_+$ is the hinge loss.
            \item $\bar{\boldsymbol{w}} = \frac{1}{T} \sum_{t = 1}^T \boldsymbol{w}_t$ is the average of the iterates
            \item $\boldsymbol{w}^* = \underset{\boldsymbol{w} \in \mathbb{R}^d}{\text{argmin}} (F(\boldsymbol{w}))$ is the optimal solution of the SVM problem
            \item $G = \underset{t=1,\dots,T}{\max} \Vert \nabla \ell_{s_t}(\boldsymbol{w}_t) \Vert$ is the norm of the loss gradient            
            \item $\lambda > 0$ is the regularization parameter
            \item $T$ is the number of iterations
        \end{itemize}
        
        We can bound $G$ bounding the norm of the loss gradient in the following way: $\Vert \nabla \ell_t(\boldsymbol{w}_t) \Vert \leq X + \lambda \Vert \boldsymbol{w}_t \Vert \leq 2X$ where $X = \max_{t = 1, \dots, m} \Vert \boldsymbol{x}_t \Vert$ is the maximum norm of the training examples.

        $$
    \mathbb{E}\left[F(\bar{\boldsymbol{w}})\right] \leq F(\boldsymbol{w}^*) + \frac{2 X^2}{\lambda T} (\ln T + 1)
        $$
\end{itemize}
