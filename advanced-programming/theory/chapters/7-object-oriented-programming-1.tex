\section{Object Oriented Programming in Python}

Classes, Inheritance \& Polymorphism

\subsection{Object-Oriented Programming}

\subsubsection{Introduction}

Python is a multi-paradigm programmin language

Many claims that:

\begin{center}
Python is object-oriented
\end{center}

Python is just \textbf{object-based} but we can use it as if it is object-oriented

Look at

\begin{center}
\textbf{Reference}
Peter Wagner
\textbf{Dimensions of Object-Based Language Design}
In Proceegings of OOPSLA'87, pp. 168-182, October 1987.
\end{center}

for the differences

\subsubsection{Wagner’s OO Taxonomy: Objects, Classes and Inheritance}

\textbf{Objects}
An object has a set of operations and a state that remembers the effect of the operations

\textbf{Class}
A class is a template from which objects may be created

\begin{itemize}
	\item object of the same class have common operations and (therefore) uniform behavior
	\item Class expose a set of operations (public interface) to its clients
\end{itemize}

\textbf{Inheritance}
A class may inherit operations from superclasses and its operations inherited by subclasses

\begin{itemize}
	\item  inheritance can be single or multiple
\end{itemize}

\subsubsection{Wagner’s OO Taxonomy (Cont.’d)}

Wagner suggests 3 classes for programming languages:

\begin{itemize}
	\item object-based = object
	\item class-based = object + classes
	\item object-oriented = onject + classes + inheritance
\end{itemize}

\textbf{Data Abstraction}
A data abstraction is an object whose state is accessible only through its operations

\begin{itemize}
	\item this concept brings forth to the data hifing property
\end{itemize}

\textbf{Delegation}
Delegation is a mechanism to delegate responsibility for performing an operation to one or more designed ancestors

\begin{itemize}
		\item note that ancestors are not always designed by inheritance in this case it is called clientship
\end{itemize}

\subsubsection{Class Definition: Rectangle}

\begin{lstlisting}[language=Python]
	class rectangle:
		def __init__(self, width, height):
			self._width=width
			self._height=height
		def calculate_area(self):
			return self._width*self._height
		def calculate_perimeter(self):
			return 2*(self._height+self._width)
		def __str__(self):
			return "I'm a Rectangle! My sides are: {0}, {1} \n My area is {2}".format(self._width,self._height, self.calculate_area())
\end{lstlisting}

\begin{lstlisting}[language=Python]
>>> from rectangle import rectangle
>>> r = rectangle(7,42)
>>> print(r)
I'm a Rectangle! My sides are: 7, 42
My area is 294
\end{lstlisting}

\subsubsection{Inheritance}

Inheritance permits to reuse and specialize a class

\textbf{Shape <-- rectangle (super class) <-- square (sub class)}

\begin{lstlisting}[language=Python]
class shape:
	def calculate_area(self): pass
	def calculate_perimeter(self): pass
	def __str__(self): pass
\end{lstlisting}

\begin{lstlisting}[language=Python]
from rectangle import rectangle

class square(rectangle):
	def __init__(self, width):
		self._width=width
		self._height=width
	def __str__(self):
		return \
			"I'm a Square! My side is: {0}\n \
			   My area is {1}".format( \
	             self._width, self.calculate_area())
\end{lstlisting}

\subsubsection{Inheritance \& Polymorphism}

\begin{lstlisting}[language=Python]
>>> from rectangle import rectangle
>>> from square import square
>>> from circle import circle
>>> shapes = [square(7), circle(3.14), rectangle(6,7), square(5),
circle(.7), rectangle(7,2), square(2)]
>>> shapes
[<square.square object at 0x80c698c>, <circle.circle object at 0x80c69ac>,
<rectangle.rectangle object at 0x80c69cc>, <square.square object at 0x80c69ec>,
<circle.circle object at 0x80c6a0c>, <rectangle.rectangle object at 0x80c6a2c>,
<square.square object at 0x80c6a4c>]
>>> for i in shapes: print(i)
...
I'm a Square! My side is: 7
My area is 49
I'm a Circle! My ray is: 3.14
My area is 30.9748469273
I'm a Rectangle! My sides are: 6, 7
My area is 42
I'm a Square! My side is: 5
My area is 25
I'm a Circle! My ray is: 0.7
My area is 1.53938040026
I'm a Rectangle! My sides are: 7, 2
My area is 14
I'm a Square! My side is: 2
My area is 4
\end{lstlisting}

\subsubsection{Inheritance \& Polymorphism Duck Typing}

... but is shape really necessary? NO

\begin{lstlisting}[language=Python]
class rectangle:
	def __init__(self, w, h):
		self._width=w
		self._height=h
def calculate_area(self):
	return \
	   self._width*self._height
def calculate_perimeter(self):
	return \
	   2*(self._height+self._width)
def __str__(self):
	return ...
\end{lstlisting}

\begin{lstlisting}[language=Python]
class circle:
	def __init__(self, ray):
		self._ray=ray
def calculate_area(self):
	return self._ray**2*math.pi
def calculate_perimeter(self):
	return 2*self._ray*math.pi
def __str__(self):
	return ...
\end{lstlisting}

\begin{lstlisting}[language=Python]
class square(rectangle): 
	def __init__(self, width):
		self._width=width
		self._height=width
	def __str__(self):
		return ...
\end{lstlisting}

\begin{lstlisting}[language=Python]
>>> from rectangle import rectangle
>>> from square import square
>>> from circle import circle
>>> shapes = [square(7), circle(3.14), rectangle(6,7), square(5),
... circle(.7), rectangle(7,2), square(2)]
>>> for i in shapes: print(i)
I'm a Square! My side is: 7
My area is 49
... 
\end{lstlisting}

\subsubsection{Summarizing}

\textbf{The meaning of class is changed}
\begin{itemize}
	\item super classes do not impose a behavior (no abstract classes or interfaces)
	\item super classes are used to group and reuse functionality
\end{itemize}

\textbf{Late binding quite useless}
\begin{itemize}
	\item no static/dynamic type
	\item duck typing
\end{itemize}

\textbf{Class vs instance members}
\begin{itemize}
	\item no real distinction between fields and methods
	\item class is just the starting point
	\item a member does not exist until you use it (dynamic typing)
\end{itemize}
