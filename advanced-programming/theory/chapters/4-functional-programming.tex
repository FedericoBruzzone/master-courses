\section{Functional programming in Python}

\subsection{Functional programming}

\subsubsection{Overvirew}

\textbf{What is functional programming?}
\begin{itemize}
	\item Functions are first class (objects).

That is, everything you can do with "data" can be done with functions themselves (such as passing a function to another function).

	\item Recursion is used as a primary control structure.

In some languages, no other "loop" construct exists.

	\item There is focus on \textbf{list processing}.

Lists are often used with recursion on sub-lists as substitute for loops.

	\item "Pure" functional languages eschew side-effects.
	
This excludes assignments to track the program state.

This discorages the use of statements in favor of expression evaluations.
\end{itemize}

\textbf{Whys}

\begin{itemize}
	\item All these characteristics make for more rapidly developed, shorter, and less bug-prone code.
	\item A lot easier to prove formal properties of functional languages and programs than of imperative languages and programs.
\end{itemize}

\subsection{Functional programming in Python}

\subsubsection{map(), filter() \& reduce()}

\textbf{Python has functional capability since its first release}
with new  releases just a few syntactical sugar has been added

\textbf{Basic elements of functional programming in python are:}
\begin{itemize}
	\item \textbf{map()}: it applies a function to a sequence.
\begin{lstlisting}[language=Python]
>>> import math
>>> print(list(map(math.sqrt, [x**2 for x in range(1,11)])))
> [1.0, 2.0, 3.0, 4.0, 5.0, 6.0, 7.0, 8.0, 9.0, 10.0]
\end{lstlisting}
	\item \textbf{filter()}: it extracts from a list those elements which verify the passed function.
\begin{lstlisting}[language=Python]
>>> def odd(x): return (x%2 != 0)
>>> print(list(filter(odd, range(1,30))))
> [1, 3, 5, 7, 9, 11, 13, 15, 17, 19, 21, 23, 25, 27, 29]
\end{lstlisting}
\item \textbf{reduce()}: it reduces a list to a single element according to the passed funcion.
\begin{lstlisting}[language=Python]
>>> import functools
>>> def sum(x,y): return x+y
>>> print(functools.reduce(sum, range(1000)))
> 499500
\end{lstlisting}
\end{itemize}

Note, \textbf{map()} and \textbf{filter()} return an iterator rather than a list.

\subsubsection{Eliminating flow control statements: if}
\textbf{Short-circuit conditional call instead of if}
\begin{lstlisting}[language=Python]
def cond(x):
	return (x==1 and 'one') or (x==2 and 'two') or 'other'

def cond2(x):
	return ('one' and x==1) or ('two' and x==2) or 'other'

if __name__ == "__main__":
	for i in range(3):
		print("cond({0}) :- {1}".format(i, cond(i)))
	for i in range(3):
		print("cond2({0}) :- {1}".format(i, cond2(i)))
\end{lstlisting}
\begin{lstlisting}[language=Python]
> cond(0) :- other
> cond(1) :- one
> cond(2) :- two

> cond2(0) :- other
> cond2(1) :- True
> cond2(2) :- True
\end{lstlisting}

\textbf{Doing some abstraction}
\begin{lstlisting}[language=Python]
block = lambda s: s
cond = lambda x: (x==1 and block('one')) or 
                 (x==2 and block('two')) or 
                 block('other')

if __name__ == "__main__":
	print("cond({0}) :- {1}".format(3, cond(3)))
\end{lstlisting}
\begin{lstlisting}[language=Python]
> cond(3) :- other
\end{lstlisting}


\subsubsection{Do abstraction: Lambda functions}
The name lambda comes from lambda-calculus which uses the greek letter lambda to represent a similar comncept.

Lambda is a term used to refer to an \textbf{anonymous function}.
\begin{itemize}
	\item that is, a block of code which can be executed as if it were a function but without a name.
\end{itemize}

Lambdas can be defined anywhere a legal expression can occur.

A lambda looks like this:
\begin{lstlisting}[language=Python]
lambda "args": "an expr on the argss"
\end{lstlisting}

Thus the previous \textbf{reduce()} example could be rewritten as:
\begin{lstlisting}[language=Python]
>>> import functools
>>> print(functools.reduce(lambda i,j: i+x, range(10000)))
> 499500
\end{lstlisting}
Alternatively the lambda can assigned to a variable as in:
\begin{lstlisting}[language=Python]
>>> add = lambda i,j: i+j
>>> print(functools.reduce(add, range(10000)))
> 499500
\end{lstlisting}

\subsubsection{Envolving factorial}

\textbf{Traditional implementation}
\begin{lstlisting}[language=Python]
def fact(n):
	return 1 if n <= 1 else n*fact(n-1)
\end{lstlisting}
\textbf{Short-circuit implementation}
\begin{lstlisting}[language=Python]
def ffact(n):
	return (n <= 1 and 1) or n*ffact(n-1)	
\end{lstlisting}
\textbf{reduce()-based implementation}
\begin{lstlisting}[language=Python]
from functools import reduce
def f2fact(p):
	return reduce(lambda n,m: n*m, range(1, p+1))
\end{lstlisting}

\subsubsection{Eliminating flow control statements: sequence}

\textbf{Sequential program flow is typical of imperative programming}
it basically relies on side-effect (variable assignments)

\textbf{This is basically in contrast with the functional approach.}

\textbf{In a list processing style we have:}
\begin{lstlisting}[language=Python]
# let's create an execution utility function
do_it = lambda f: f()
# let f1, f2, f3 (etc) be functions that perform actions
map(do_it, [f1, f2, f3])
\end{lstlisting}
\begin{itemize}
	\item single statements of the sequence are replaced by funcions
	\item the sequene is realized by mapping an activation function to all the function objects that should compose the sequence.
\end{itemize}

\subsubsection{Eliminating while statements: Echo}

\textbf{Statement-based echo function}
\begin{lstlisting}[language=Python]
def echo_IMP():
	while True:
		x = input("FP -- ")
		if x == 'quit': break
		else: print(x)
		
if __name__ == "__main__": echo_IMP()
\end{lstlisting}

\textbf{First step toward a functional solution}
\begin{itemize}
	\item No print
	\item Utility function for "identity with side-effect" (a monad)
\end{itemize}
\begin{lstlisting}[language=Python]
def monadic_print(x)
	print(x)
	return x
\end{lstlisting}

\textbf{Functional version of the echo function}
\begin{lstlisting}[language=Python]
echo_FP = 
	lambda : monodic_print(input("FP -- ")=='quit' or echo_FP())

if __name__ == "__main__": echo_FP()
\end{lstlisting}

\subsubsection{Whys}

\textbf{Why? To eliminate the side-effects}
\begin{itemize}
	\item mostly all errors depend on variables that obtain unexpected values.
	\item functional programs bypass the issue by not assigning values to variables at all.
\end{itemize}

\textbf{E.g., To determine the pairs whose product is >25}
\begin{lstlisting}[language=Python]
def bigmuls(xs,ys):
	bigmuls = []
	for x in xs:
		for y in ys:
			if x*y > 25:
				bigmuls.append((x,y))
	return bigmuls
	
if __name__ == "__main__":
print(bigmuls((1,2,3,4),(10,15,3,22)))
\end{lstlisting}

 
\begin{lstlisting}[language=Python]
from functools import reduce
import itertools

bigmuls = lambda xs, ys: [x_y for x_y in
						  combine(xs,ys) if x_y[0]*x_y[1] > 25]
						  
combine = lambda xs, ys: itertools.zip_longest(
			             xs*len(ys), dupelms(ys,len(xs)))
			             
dupelms = lambda lst, n: reduce(lambda s, t: s+t,
                                list(map(lambda l,n=n: [l]*n, lst)))
                                
if __name__ == "__main__":
print(bigmuls([1,2,3,4],[10,15,3,22]))
\end{lstlisting}

\subsubsection{Future of map(), reduce() \& filter()}
\textbf{The future of the python's map(), filter(), and reduce is uncertain.}

\textbf{Comprehensions can easily replace map() and filter()}

\begin{itemize}
	\item \textbf{map()} can be replaced by
\begin{lstlisting}[language=Python]
>>> import math
>>> [math.sqrt(x**2) for x in range(1,11)]
> [1.0, 2.0, 3.0, 4.0, 5.0, 6.0, 7.0, 8.0, 9.0, 10.0]
\end{lstlisting}
	\item \textbf{filter()} can be replaced by
\begin{lstlisting}[language=Python]
>>> def odd(x): return (x%2 != 0)
>>> [x for x in range(1,30) if odd(x)]
[1, 3, 5, 7, 9, 11, 13, 15, 17, 19, 21, 23, 25, 27, 29]
\end{lstlisting}
\end{itemize}

Guido von Rossum finds the \textbf{reduce()} too cryptic and prefers to use more ad hoc functions instead
\begin{itemize}
	\item \textbf{sum()}, \textbf{any()} and \textbf{all()}
\end{itemize}
To have moved \textbf{reduce()} in a module in Python 3 should render manifest his intent.