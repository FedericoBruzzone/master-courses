\section{Modularizing Pythons}

Modules \& Packages

\subsection{Modules}

\subsubsection{Basics on Modules}

A \textbf{module} is a simple text file of Python's statements

\begin{lstlisting}[language=Python]
import <<module name>>
\end{lstlisting}

Lets a client (importer) fetch a module as a whole

\begin{lstlisting}[language=Python]
from <<module name>> import <<name list>>|*
\end{lstlisting}

Allows clients to fetch particular names from a module

\begin{lstlisting}[language=Python]
imp.reload <<module name>>
\end{lstlisting}

Provides a way to reload a module’s code without stopping the Python interpreter

\subsubsection{How import work}

Import are run-time operations that:

\begin{itemize}
  \item find the module's file
  \begin{lstlisting}[language=Python]
  import rectangle
  \end{lstlisting}
  This looks for rectangle.py though a standard module search path
  \item load module’s bytecode (from a .pyc file named after the module)
  - if a source file newer than the .pyc is found or
  - no bytecode is found the module source file is compiled
  compilation occurs when the module is imported
  -so only imported modules will leave a .pyc file on your system
  \item run the module’s code to build the objects it defines.
\end{itemize}

\subsubsection{Python’s Module Search Path}

\begin{itemize}
  \item The current directory
  \item PYTHONPATH directory (if set)
  \item Standard library directories
  \item The content of any .pth files
\end{itemize}

\subsubsection{Imports Happen Only Once}

Modules are imported only once, so, code is executed just at import time.

Let us consider:
\begin{center}
  \includegraphics[scale=0.7]{7-import-open-only-once}
\end{center}

- the module simple is imported just the first time
- the assignment to spam in the module is only executed the first time

\subsubsection{import and from Are Assignments}

- they may be used in statements, in function definition, ...;
- they are not resolved or run until the execution flow reaches them.

import and from are assignement
\begin{lstlisting}[language=Python]
x = 1
y = [1, 2]
\end{lstlisting}

import assigns an entire module object to a single name

\begin{lstlisting}[language=Python]
[23:10]cazzola@hymir:~/esercizi-pa>python3
>>> import small
>>> small
<module ’small’ from ’small.py’>
\end{lstlisting}

from assigns new names to homonyms objects of another module

\begin{lstlisting}[language=Python]
>>> from small import x, y
>>> x = 42
>>> y[0] = 42
>>> import small
>>> small.x
1
>>> small.y
[42, 2]
\end{lstlisting}

\subsubsection{"import" and "form" equivalence}

The following
\begin{lstlisting}[language=Python]
from small import x,y
\end{lstlisting}

is equivalent to
\begin{lstlisting}[language=Python]
import small
x = small.x
y = small.y
del small
\end{lstlisting}

\begin{lstlisting}[language=Python]
[9:03]cazzola@hymir:~/esercizi-pa>python3
>>> from small import x,y
>>> small
Traceback (most recent call last):
File "<stdin>", line 1, in <module>
NameError: name ’small’ is not defined
>>> import small
>>> small
<module ’small’ from ’small.py’>
>>> x = small.x
>>> y = small.y
>>> del small
>>> small
Traceback (most recent call last):
File "<stdin>", line 1, in <module>
NameError: name ’small’ is not defined
>>> x
1
\end{lstlisting}

\subsubsection{Module namespace}
Files generate namespaces

\begin{itemize}
   \item module statements run once at the first import
   \item every name that is assigned to a value at the top level of a module file becomes one of its attributes
   \item module namespaces can be accessed via \_\_dict\_\_ or dir(module)
   \item  module are single scope (i.e., local is global)
\end{itemize}


\begin{lstlisting}[language=Python]
print(’starting to load...’)
import sys
name = 42
def func(): pass
print(’done loading’)
\end{lstlisting}

\begin{lstlisting}[language=Python]
[23:37]cazzola@hymir:~/esercizi-pa>python3
>>> import module2
starting to load...
done loading
>>> module2.sys
<module ’sys’ (built-in)>
>>> module2.name
42
>>> module2.func
<function func at 0xb7a0cbac>
>>> list(module2.__dict__.keys())
[’name’,’__builtins__’,’__file__’,’__package__’,’sys’,’func’,’__name__’,’__doc__’]
\end{lstlisting}

\subsubsection{Module reload}

The imp.reload function forces an already loaded module’s code to be reloaded and rerun

- Assignments in the file’s new code change the existing module object in-place

\begin{lstlisting}[language=Python]
# changer.py
message = "First version"
def printer():
print(message)
\end{lstlisting}

\begin{lstlisting}[language=Python]
>>> import changer
>>> changer.printer()
First version
>>> ^Z
Suspended
[9:57]cazzola@hymir:~/esercizi-pa>gvim changer.py
[9:58]cazzola@hymir:~/esercizi-pa>fg
python3
\end{lstlisting}

\begin{lstlisting}[language=Python]
# changer.py after the editing
message = "After editing"
def printer():
print(’reloaded:’, message)
\end{lstlisting}

\begin{lstlisting}[language=Python]
>>> import changer
>>> changer.printer()
First version
>>> from imp import reload
>>> reload(changer)
<module ’changer’ from ’changer.py’>
>>> changer.printer()
reloaded: After editing
\end{lstlisting}

\subsection{Packages}

\subsubsection{Basics on Python’s Packages}

An import can name a directory path.
\begin{itemize}
  \item A directory of Python code is said to be a package, so such imports are known as package imports.
  \item A package import turns a directory into another Python namespace, with attributes corresponding to the subdirectories and module files that the directory contains. 
\end{itemize}

Packages are organized in directories, e.g., dir0/dir1/mod0

























