\section{Metaclasses}

How to Silently Extend Classes (Part 3)

\subsection{Metaclasses}

\subsubsection{What’s a Metaclass?}

Metaclasses are a mechanism to gain a high-level of control over how a set of classes works.

\begin{itemize}
	\item They permit to intercept and augment class creation;
	\item they provide an API to insert extra-logic at the end of class statement
	\item they provide a general protocol to manage class objects in a program
\end{itemize}

Note,

\begin{itemize}
	\item the added logic does not rebind the class name to a decorator callable, but rather routes creation of the class itself to specialized logic
	\item metaclasses add code to be run at class creation time and not at instance creation time
\end{itemize}

\subsubsection{The Metaclass Model}

Classes Are Instances of type

\begin{lstlisting}[language=Python]
[11:44]cazzola@hymir:~/esercizi-pa>python3
>>> from circle import *
>>> type(circle)
<class ’type’>
>>> circle.__class__
<class ’type’>
>>> c = circle(3)
>>> type(c)
<class ’circle.circle’>
>>> c.__class__
<class ’circle.circle’>
>>> type([])
<class ’list’>
>>> type(type([]))
<class ’type’>
\end{lstlisting}

\textbf{Metaclasses Are Subclasses of type}

\begin{itemize}
	\item type is a class that generates user-defined classes.
	\item Metaclasses are subclasses of the type class
	\item Class objects are instances of the type class, or a subclass thereof
	\item Instance objects are generated from a class.
\end{itemize}

\textbf{Class Statement Protocol}

\begin{itemize}
	\item at the end of class statement, after filling \_\_dict\_\_, python calls
		class = type(classname, superclasses, attributedict)
	class = type(classname, superclasses, attributedict)
	\item type object defines a \_\_call\_\_ operator that calls \_\_new\_\_ (to create class objects) and \_\_init\_\_ (to create instance objects) when type object is called
\end{itemize}

\subsubsection{The Metaclass Declaring \& Coding}

\textbf{To create a class with a custom metaclass you have just to list the desired metaclass as a keyword argument in the class header.}

\begin{lstlisting}[language=Python]
class Spam(metaclass=Meta): pass
\end{lstlisting}

\textbf{Coding Metaclasses}
\begin{itemize}
	\item subtype type
	\item override \_\_new\_\_, \_\_init\_\_ and \_\_call\_\_ operators
\end{itemize}

\subsubsection{The Metaclass Declaring & Coding (Cont’d)}

\begin{lstlisting}[language=Python]
class Meta(type):
	def __new__(cls, classname, supers, classdict):
		print(f"[__new__] cls :- {cls}\n
			[__new__] classname :- {classname}\n
			[__new__] supers :- {supers}\n
			[__new__] classdict :- {classdict}")
		print(f"[__new__] dict :- {cls.__dict__}")
		return type.__new__(cls, classname, supers, classdict)
	def __init__(cls, classname, supers, classdict):
		print(f"[__init__] cls :- {cls}\n
			[__init__] classname :- {classname}\n
			[__init__] supers :- {supers}\n
			[__init__] classdict :- {classdict}")
		print(f"[__init__] dict :- {cls.__dict__}")
		type.__init__(cls, classname, supers, classdict)
	def __call__(cls, arg):
		print(f"[__call__] cls :- {cls}\n
			[__call__] arg :- {arg}")
		return type.__call__(cls, arg)
		
print("### Making Classes...")
class Egg: pass

class Spam(Egg, metaclass=Meta):
	data = 42
	def __init__(self, string):
		self.string = string
	def meth(self, arg):
		print(f"### [meth] arg :- {arg}\t
			data :- {self.data}")

print("### Making Instances...")
X = Spam("Hello World")
X.meth(42)
\end{lstlisting}

\begin{lstlisting}[language=Python]
[23:01]cazzola@mangog:~>python3 metazero.py
### Making Classes...
[__new__] cls :- <class ’__main__.Meta’>
[__new__] classname :- Spam
[__new__] supers :- (<class ’__main__.Egg’>,)
[__new__] classdict :- {’__module__’: ’__main__’,
’__qualname__’: ’Spam’, ’data’: 42,
’__init__’: <function Spam.__init__ at 0x...>,
’meth’: <function Spam.meth at 0x...>}
[__new__] dict :- {’__module__’: ’__main__’,
’__new__’: <staticmethod(<function Meta.__new__ at 0x...>)>,
’__init__’: <function Meta.__init__ at 0x...>,
’__call__’: <function Meta.__call__ at 0x...>,
’__doc__’: None}
[__init__] cls :- <class ’__main__.Spam’>
[__init__] classname :- Spam
[__init__] supers :- (<class ’__main__.Egg’>,)
[__init__] classdict :- {’__module__’: ’__main__’,
’__qualname__’: ’Spam’, ’data’: 42,
’__init__’: <function Spam.__init__ at 0x...>,
’meth’: <function Spam.meth at 0x...>}
[__init__] dict :- {’__module__’: ’__main__’,
’data’: 42,
’__init__’: <function Spam.__init__ at 0x...>,
’meth’: <function Spam.meth at 0x...>,
’__doc__’: None}
### Making Instances...
[__call__] cls :- <class ’__main__.Spam’>
[__call__] arg :- Hello World
### [meth] arg :- 42 data :- 42
\end{lstlisting}

\subsubsection{Metaclasses vs Superclasses}

\begin{itemize}
	\item Metaclasses inherit from the type class
	\item Metaclass declarations are inherited by subclasses
	\item Metaclass attributes are not inherited by class instances
\end{itemize}

\begin{lstlisting}[language=Python]
class MetaOne(type):
	def __new__(meta, classname, supers, classdict): # Redefine type method
		print(’In MetaOne.new:’, classname)
		return type.__new__(meta, classname, supers, classdict)
	def toast(self):
		print(’toast’)

class Super(metaclass=MetaOne): # Metaclass inherited by subs too
	def spam(self): # MetaOne run twice for two classes
	print(’spam’)

class C(Super): # Superclass: inheritance versus instance
	def eggs(self): # Classes inherit from superclasses
		print(’eggs’) # But not from metclasses
		
X = C()
X.eggs() # Defined in C
X.spam() # Inherited from Super
X.toast() # Not inherited from metaclass
\end{lstlisting}

\subsubsection{Metaclass-Based Augmentation}

\begin{lstlisting}[language=Python]
def eggsfunc(obj): return obj.value * 4
def hamfunc(obj, value): return value + ’ham’

class Extender(type):
	def __new__(meta, classname, supers, classdict):
		classdict[’eggs’] = eggsfunc
		classdict[’ham’] = hamfunc
		return type.__new__(meta, classname, supers, classdict)

class Client1(metaclass=Extender):
	def __init__(self, value): self.value = value
	def spam(self): return self.value * 2

class Client2(metaclass=Extender): value = ’ni?’

X = Client1(’Ni!’)
print(X.spam())
print(X.eggs())
print(X.ham(’bacon’))
Y = Client2()
print(Y.eggs())
print(Y.ham(’bacon’))
\end{lstlisting}

\begin{lstlisting}[language=Python]
[18:01]cazzola@hymir:~/esercizi-pa/metaclass>python3 extender.py
Ni!Ni!
Ni!Ni!Ni!Ni!
baconham
ni?ni?ni?ni?
baconham
\end{lstlisting}

\subsubsection{Applying Decorators to Methods: The Decorators}

\begin{lstlisting}[language=Python]
# timer.py
import time
def timer(label=’’, trace=True):
	def onDecorator(func):
		def onCall(*args, **kargs):
			start = time.clock()
			result = func(*args, **kargs)
			elapsed = time.clock() - start
			onCall.alltime += elapsed
			print(’{0}{1}: {2:.5f}, {3:.5f}’.format(
				label, func.__name__, elapsed, onCall.alltime))
			return result
		onCall.alltime = 0
	return onCall
return onDecorator
\end{lstlisting}

\begin{lstlisting}[language=Python]
# tracer.py
def tracer(func):
	calls = 0
	def onCall(*args, **kwargs):
		nonlocal calls
		calls += 1
		print(’call {0} to {1}’.\
			format(calls, func.__name__))
	return func(*args, **kwargs)
return onCall
\end{lstlisting}

\subsubsection{Applying Decorators to Methods: The Decorators}

\begin{lstlisting}[language=Python]
from decorators.tracer import tracer
class Person:
	@tracer
	def __init__(self, name, pay):
		self.name = name
		self.pay = pay
	@tracer
	def giveRaise(self, percent): # giveRaise = tracer(giverRaise)
		self.pay *= (1.0 + percent) # onCall remembers giveRaise
	@tracer
	def lastName(self): # lastName = tracer(lastName)
		return self.name.split()[-1]

bob = Person(’Bob Smith’, 50000)
sue = Person(’Sue Jones’, 100000)
print(bob.name, sue.name)
sue.giveRaise(.10) # Runs onCall(sue, .10)
print(sue.pay)
print(bob.lastName(), sue.lastName()) # Runs onCall(bob), remembers lastName
\end{lstlisting}

\begin{lstlisting}[language=Python]
[21:30]cazzola@hymir:~/esercizi-pa/metaclass>python3 Person1.py
call 1 to __init__
call 2 to __init__
Bob Smith Sue Jones
call 1 to giveRaise
110000.0
call 1 to lastName
call 2 to lastName
Smith Jones
\end{lstlisting}

\subsubsection{Applying Decorators to Methods: Through a Metaclass!}

\begin{lstlisting}[language=Python]
from types import FunctionType
from decorators.tracer import tracer

class MetaTrace(type):
	def __new__(meta, classname, supers, classdict):
		for attr, attrval in classdict.items():
			if type(attrval) is FunctionType:
				classdict[attr] = tracer(attrval)
		return type.__new__(meta, classname, supers, classdict)

class Person(metaclass=MetaTrace):
	def __init__(self, name, pay):
		self.name = name
		self.pay = pay
	def giveRaise(self, percent):
		self.pay *= (1.0 + percent)
	def lastName(self): return self.name.split()[-1]

bob = Person(’Bob Smith’, 50000)
sue = Person(’Sue Jones’, 100000)
print(bob.name, sue.name)
sue.giveRaise(.10)
print(sue.pay)
print(bob.lastName(), sue.lastName())
\end{lstlisting}

\begin{lstlisting}[language=Python]
[21:45]cazzola@hymir:~/esercizi-pa/metaclass>python3 Person2.py
call 1 to __init__
call 2 to __init__
Bob Smith Sue Jones
call 1 to giveRaise
110000.0
call 1 to lastName
call 2 to lastName
Smith Jones
\end{lstlisting}

\begin{lstlisting}[language=Python]
from types import FunctionType
from decorators.tracer import tracer
from decorators.timer import timer
def decorateAll(decorator):
	class MetaDecorate(type):
		def __new__(meta, classname, supers, classdict):
			for attr, attrval in classdict.items():
				if type(attrval) is FunctionType:
					classdict[attr] = decorator(attrval)
			return
				type.__new__(meta,classname,supers,classdict)
		return MetaDecorate
class Person(metaclass=decorateAll(tracer)):
...
print(’--- tracer’)
bob = Person(’Bob Smith’, 50000)
sue = Person(’Sue Jones’, 100000)
print(bob.name, sue.name)
sue.giveRaise(.10)
print(sue.pay)
print(bob.lastName(), sue.lastName())
class Person(
metaclass=decorateAll(timer(label=’**’))):
...
print(’--- timer’)
bob = Person(’Bob Smith’, 50000)
sue = Person(’Sue Jones’, 100000)
print(bob.name, sue.name)
sue.giveRaise(.10)
print(sue.pay)
print(bob.lastName(), sue.lastName())
print(’{0:.5f}’.format(Person.__init__.alltime))
print(’{0:.5f}’.format(Person.giveRaise.alltime))
print(’{0:.5f}’.format(Person.lastName.alltime))
\end{lstlisting}

\begin{lstlisting}[language=Python]
[21:47]cazzola@hymir:~/esercizi-pa/metaclass>python3 Person3.py
--- tracer
call 1 to __init__
call 2 to __init__
Bob Smith Sue Jones
call 1 to giveRaise
110000.0
call 1 to lastName
call 2 to lastName
Smith Jones
--- timer
**__init__: 0.00000, 0.00000
**__init__: 0.00000, 0.00000
Bob Smith Sue Jones
**giveRaise: 0.00000, 0.00000
110000.0
**lastName: 0.00000, 0.00000
**lastName: 0.00000, 0.00000
Smith Jones
0.00000
0.00000
0.00000
\end{lstlisting}



