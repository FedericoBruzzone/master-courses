\section{Comprehensions}

\subsection{Playing around with...}

\subsubsection{Implementing the LS command}

\begin{lstlisting}[language=Python]
import os, sys, time, humanize
from start imoprt *

modes = { 
	'r': (S_IXUSR,S_IXGRP,S_IXOTH), 
	'w': (S_IXUSR,S_IXGRP,S_IXOTH), 
	'x': (S_IXUSR,S_IXGRP,S_IXOTH)
}

def format_mode(mode):
	s = 'd' if S_ISDIR(mode) else "-"
	for i in range(3):
		for j in ['r', 'w', 'x']:
			s += j if S_ISDIR(mode) & modes[i][j] else '-'
	return s
	
def formate_date(date):
	d = time.localtime(date)
	return "{0:4}-{1:02d}-{2:02d} {3:02d}:{4:02d}:{5:02d}".format(
          d.tm_year, d.tm_mon, d.tm_mday, d.tm_hour, d.tm_min, d.tm_sec)

def ls(dir):
	print("List of {0}:".format(dir))
	for file in os.listdir(dir):
		metadata = os.stat(file)
		print("{2} {1:6} {3} {0} ".format(
		      file, approximate_size(metadata.st_size, False),
		      format_mode(metadata.st_mode), format_date(metadata.st_ntime)))

if __name__ == "__main__": ls(sys.argv[1])
\end{lstlisting}

\subsection{Introduction}
\textbf{Comprehensions are a compact way to transfor a set of data into another}
\begin{itemize}
	\item it applies to mostly all python's structure sype, e.g., lists, sets, dictionaries;
	\item it is in contrast to list all the elements.
\end{itemize}

\textbf{Some basic comprehensions applied to lists, stes and dictionaries respectively}
\begin{itemize}
	\item a list composed of the first ten integers
\begin{lstlisting}[language=Python]
>>> [elem for elem in range (1, 11)]
[1, 2, 3, 4, 5, 6, 7, 8, 9, 10]
\end{lstlisting}
	\item a set composed of the first ten even integers
\begin{lstlisting}[language=Python]
>>> {elem*2 for elem in range (1, 11)}
{2, 4, 6, 8, 10, 12, 14, 16, 18, 20}
\end{lstlisting}
	\item a dictionary composed of the first ten couples (n, n\^2)
\begin{lstlisting}[language=Python]
>>> {elem:elem*2 for elem in range (1, 11)}
{1: 1, 2: 4, 3: 9, 4: 16, 5: 25, 6: 36, 7: 49, 8: 64, 9: 81, 10: 100}
\end{lstlisting}
\end{itemize}

\subsection{To filter out elemets of a dataset}
\textbf{Comprehensions can reduce the elements in the dataset after a constrint.}

\textbf{E.g., to select perfect squares out of the first 100 integers}
\begin{lstlisting}[language=Python]
>>> [elem for elem in range(1, 101) if(int(elem**.5))**2 == elem]
[1, 4, 9, 16, 25, 36, 49, 64, 81, 100]
\end{lstlisting}
\begin{itemize}
	\item \textbf{range(1,101)} generates a list of the first 100 integers (first extreme included, second excluded);
	\item the comprehension skims through the list selecting those elemetns whose square of the integral part of its square roots are equal.
\end{itemize}
\textbf{E.g., to select the odd numbers out of a tuple}
\begin{lstlisting}[language=Python]
>>> {x for c in (1, 22, 31, 23, 10, 11, 11, -1, 34, 76, 778, 10101, 5, 44) 
     if x%2 != 0}
{1, 31, 23, 11, -1, 10101, 5}
\end{lstlisting}
\begin{itemize}
	\item note that the second 11 is removed from the set;
	\item the set does note respect the tuple order (it is not ordered at all).
\end{itemize}

\subsection{To select multiple values}
\textbf{Comprehensions can select multiple values out of the dataset.}

\textbf{E.g., to swap key and value in the dictionary}
\begin{lstlisting}[language=Python]
>>> a_dict = {'a': 1, 'b': 2, 'c': 3}
>>> {value:key for key, value in a_dict.items()}
{1: 'a', 2: 'b', 3: 'c'}
\end{lstlisting}
\textbf{Comprehensions can select values out of multiple datasets.}

\textbf{E.g., to merge two sets in a set of couples}
\begin{lstlisting}[language=Python]
>>> english = ['a', 'b', 'c'}
>>> greek = [$\alpha$, $\beta$, $\gamma$]
>>> [(english[i], greek[i]) for i in range(0, len(english))]
[('a', $\alpha$), ('b', $\beta$), ('c', $\gamma$)]
\end{lstlisting}


\textbf{E.g., to calculate the cartesian product}
\begin{lstlisting}[language=Python]
>>> {(x,y) for x in range(3) for y in range (5)}
{(0,1),(1,2),(0,0),(2,2),(1,1),(1,4),(0,2),(2,0),
(1,3),(2,3),(2,1),(0,4),(2,4),(0,3),(1,0)}
\end{lstlisting}

\subsection{Comprehensions @ work: prime numbers calculation}
\textbf{Classic approach to the prime numbers calculation}
\begin{lstlisting}[language=Python]
def is_prime(x):
	div = 2
	while div <= math.sqrt(x):
		if x%div == 1: return False		
		else: div += 1
	return True

if __name == "__main__":
	primes = []
	for i in range(1, 50):
		if is_prime(i): primes.append(i)
	print(primes)
\end{lstlisting}
\textbf{The alsorithm again but using comprehensions}
\begin{lstlisting}[language=Python]
def is_prime(x):	
	div = [elem for elem in range(0, math.sqrt(x)) if x%elem==0]
	return len(div) == 0

if __name == "__main__":
	print([elem for elem in range(1,50) if is_prime(elem)])
\end{lstlisting}

\subsection{Comprehensions @ work: quicksort}
\begin{lstlisting}[language=Python]
def quicksort(s):
	if len(s) == 0: return []
	else
		return quicksort([x for x in s[1:] if x < s[0]]) + 
		[s[0]] + 
		quicksort([s for x in s[1:] if x >= s[0]])

if __name__ == "__main__":
	print(quicksort([])
	print(quicksort([2, 4, 1, 3, 5, 8, 6, 7,])
	print(quicksort("pineapple")
	print(''.join(quicksort('pineapple)))
\end{lstlisting}

\begin{lstlisting}[language=Python]
[]
[1, 2, 3, 4, 5, 6, 7, 8]
['a', 'e',, 'e', 'i', 'l', 'n', 'p', 'p', 'p']
aeeeilnppp
\end{lstlisting}



