\section{Primitive Datatypes \& recursion in Python}

\subsection{Python's Native Datatypes}

\subsubsection{Introduction}
In python \textbf{every value has a datatype}, but you do not need to declare it.

\textbf{How does that work?}

Based on each variable's assignment, python figures out what type it is and keeps tracks of that internally.

\subsubsection{Boolean}
Python provides two constants

- \textbf{True} and \textbf{False}

\textbf{Operations on Booleans}

Logic operations: \textit{and} \textit{or} \textit{not}

Relational operators: \textit{==} \textit{!=} \textit{<} \textit{>} \textit{<=} \textit{>=}

Note that python allows chains of comparisons
\begin{lstlisting}[language=Python]
>>> x = 3
>>> 1<x<=5
True
\end{lstlisting}

\subsubsection{Number}
\textbf{Two kinds of number: integer and floats}
\begin{itemize}
	\item no class declaration to distinguish them
	\item they can be distiguished by the presence/absence of the decimal point
\begin{lstlisting}[language=Python]
>>> type(1)
<class 'int'>
>>> isinstance(1, int)
True
>>> 1+1
2
>>> 1+1.0
2.0
>>>type(2.0)
<class 'float'>
\end{lstlisting}
	\item \textbf{type()} function provides the type of any value or variable;
	\item \textbf{isinstance()} check if a value or variable is of a given type;
	\item adding an int to an yields another int but adding it to a float yields a float.
\end{itemize}