\section{Teoria della complessità di approssimazione}

In questa sezione si affrontano alcune tematiche, in particolare il teorema PCP, 
con il quale di riuscirà a dimostrare l'inapprossimabilità di MaxEkSat e Indipendent set.

\subsection{Verificatori}
Il concetto di verificatore assume diverse sfaccettature nel corso 
delle sezioni che seguono, ma sostanzialmente è una macchina che si occupa di verificare 
la decisione di un certo problema per un certo input.

\subsubsection{NP attravero testimoni}
Una Macchina di Turing verificatore per un problema di decisione $L \subseteq 2^*$, riceve due input, $x$ e $w$, testimone, 
e si comporta come segue:
\begin{enumerate}
    \item se $x \notin L$ risponde no qualunque sia $w$
    \item se $x \in L$, allora $\exists w\in 2^*$ tale che la macchina risponde sì
\end{enumerate}
Devono valere però:
\begin{enumerate}
    \item $|w| \leq P(|x|)$ ovvero la lunghezza di $w$ è polinomiale nell'input
    \item la macchina lavora in tempo polinomiale
\end{enumerate}

\begin{remark}
    Si può pensare alla classe NP come una classe di problemi facilmente verificabili, ma difficilmente 
    risolvibili.
\end{remark}
\begin{remark}
    Esistono problemi nè risolvibili nè  verificabili facilmente.
\end{remark}
\begin{theorem}
    Un problema $L \subseteq 2^* \in $NP se e solo se esiste una Macchina di Turing deterministica $V$, verificatore,
    e un polinomio $P()$, tali che:
    \begin{enumerate}
        \item $V(x,w)$ lavora in tempo polinomiale in $|x|$
        \item $\forall x \in 2^*$:
        \begin{enumerate}
            \item se $x \in L$, $\exists w \in 2^*$ tale che $|w| \leq P(|x|)$ e $V(x,w) = \mathit{yes}$
            \item se $x \notin L$, $\forall w \in 2^*$ $|w| \leq P(|x|)$, $V(x,w) = \mathit{no}$
        \end{enumerate}
    \end{enumerate}
\end{theorem}

\subsubsection{NP con oracolo}
L'idea di una Macchina di Turing con oracolo presuppone l'esistenza, oltre 
al nastro classico, di un nastro dell'oracolo.

La Macchina può entrare in uno stato di query all'oracolo, 
in cui lo interroga e l'oracolo risponderà, facendole cambiare stato di conseguenza.

\begin{theorem}
    Un problmema $L \subseteq 2^* \in $ NP se esiste una Macchina di Turing con oracolo $V$ e
    un polinomip $P()$, tale che:
    \begin{enumerate}
        \item $V(x)$ lavora in tempo polinomiale in $|x|$
        \item $V(x)$ effettua al più $P(|x|)$ query all'oracolo
        \item $\forall x \in 2^*$:
        \begin{enumerate}
            \item se $x \notin L$, $V(x) = no$ qualunque sia l'oracolo
            \item se $x \in L$, $V(x) = yes$ per una qualche stringa dell'oracolo
        \end{enumerate}
    \end{enumerate}
\end{theorem}

\subsubsection{Verificatore probabilistico}
In questo caso si estende il concetto di verificatore con 
oracolo aggiungendo un nastro di numero casuali che la Macchina può leggere.

Date due funzioni 
$$q,r : \mathbb{N} \longrightarrow \mathbb{N}$$
$PCP[r,q]$ è la classe di linguaggi $L \subseteq 2^*$ per i quali esiste un verificatore probabilistico
con le seguenti proprietà: 
\begin{enumerate}
    \item $V(x)$ lavora in tempo polinomiale
    \item $V(x)$ fa al massimo $q(|x|)$ query all'oracolo
    \item $V(x)$ usa al più $r(|x|)$ bit casuali
    \item $\forall x \in 2^*$:
    \begin{enumerate}
        \item Se $x \notin L$, $V(x) = no$ con probabilità $\geq \frac{1}{2}$
        \item Se $x \in L$, $V(x) = yes$ con probabilità $1$ per almeno
        un $w \in 2^*$
    \end{enumerate}
\end{enumerate}

\begin{remark}
    Vale che:
    \begin{itemize}
        \item $NP = PCP[0,\mathit{Poly}] = \cup_{p : \mathit{polinomio}}PCP[0, p]$ 
        \item $P = PCP[0,0]$
    \end{itemize}
\end{remark}

\subsubsection{Teorema PCP}
Definiti i verificatori probabilistici, si passa ora al teorema PCP.

\begin{theorem}
    $PCP[O(\log n), O(1)] = NP$
\end{theorem}

\subsubsection{Verificatore NP in forma canonica}
Si considera ora la classe 
$$PCP[r(n), q], r(n) = O(\log n), q \in \mathbb{N}$$
Un verificatore facente parte di questa classe, può leggere 
fino a $r(|x|)$ bit random, e $q$ query all'oracolo.

Il processo seguito da un verificatore del genere è quello 
di interrogare l'oracolo e agire di conseguenza.
In particolare, si può pensare a un albero contenente le varie strade che può prendere 
il verificatore in base alla risposta dell'oracolo. Ha quindi un comportamento 
adattivo.

Quello che può fare però il verificatore, invece che chiedere di volta in volta 
all'oracolo, è simulare tutti i rami di decisione, e segnarsi quando dovrebbe 
interrogarlo, in modo da poi richiedere in una volta sola tutto quello di 
cui ha bisogno.

Un verificatore in forma canonica è proprio questo, ovvero un verificatore che prima
simula tutti i rami di computazione possibili, che variano in base all'interrogazione
all'oracolo, per poi interrogarlo su tutte le posizioni di cui ha bisogno.\\
Il verificatore così ottenuto può essere considerato come una semplice Macchina di Turing, 
infatti non deve più interrogare l'oracolo.

\subsection{Problemi inapprossimabili}
In questa sezione si affrontano due dimostrazioni di inapprossimabilità, 
esse sfruttano i concetti introdotti a sezione precedente.

\subsubsection{Inapprossimabilità MaxEeSat}
Il problema è stato introdotto a sezione \ref{eksat}, e si era individuato un algoritmo 
deterministico che sfruttava concetti probabilistici.

\begin{theorem}
    Esiste un algoritmo deterministico che approssima MaxE3Set con tasso
    $\frac{8}{7}$
\end{theorem}

\begin{theorem}
    \label{th1ek}
    Esiste un $\bar{\epsilon} > 0$ tale che MaxE3Sat non è $(\frac{8}{7}-\bar{\epsilon})$-approssimabile. 
\end{theorem}

Un versione rilassata del teorema appena introdotto è la seguente.
\begin{theorem}
    Esiste un $\bar{\epsilon} > 0$ tale che MaxE3Sat non è $(1+\bar{\epsilon})$-approssimabile. 
\end{theorem}
\begin{remark}
    Il teorema \ref{th1ek} dimostra che l'algoritmo individuato a sezione \ref{eksat} è ottimo, 
    mentre quello successivo che MaxE3Sat non fa parte di PTAS. La dimostrazione di quest'ultimo è
    molto più semplice e usa lo stesso principio.
\end{remark}
\begin{lemma}
    Esiste un $\epsilon > 0$ tale che MaxSat non è $(1+\bar{\epsilon})$-approssimabile in tempo polinomiale.
\end{lemma}
\begin{proof}
    Considerando quanto detto su PCP a sezione precedente: 
    $$L \in NP \implies \exists r(n) \in O(\log n), \exists w \in \mathbb{N}, L \in PCP[r(n), q]$$
    Significa quindi che esiste un verificatore probabilistico $V$ che opera come segue:
    $$x \in 2^* \longrightarrow V (R \in 2^{r(|x|)}, (w_{i_1}, \dots, w_{i_q})) \longrightarrow \mathit{yes}/\mathit{no}$$
    Dove $(w_{i_1}, \dots, w_{i_q})$ sono funzione di $x$ e di $R$.
    
    Fissando $x$ e $R$ si può immaginare una funzione che opera così:
    $$f^R(w_{i_1}^R, \dots, w_{i_q}^R) = \mathit{true\;sse\;V\;accetta}$$
    La funzione può essre espressa come formula sat, e ridotta a 3-sat:
    $$\Phi_x = \wedge_{R\in 2^{r(|x|)}} \phi_z^R$$
    Dove $\phi_z^R$ è la traduzione di $f^R$ in una formula logica dove le 
    variabili rappresentano il seguente evento:
    \[
        x_i = \begin{cases}
            \mathit{true\;se\;} w_i = 1\\
            \mathit{false\;se\;} w_i = 0
        \end{cases}\]

    $\Phi_x$ ha $2^{r(|x|)} \cdot 2^q$ clausole, ovvero una quantità polinomiale perch
    $r(|x|) \in O(\log n)$.

    Distinguiamo ora due casi:
    \begin{enumerate}
        \item se $x \in L$ esiste un $w$ per cui $V$ accetta con probabilità 1, 
        ovvero per ogni $R \in 2^{r(|x|)} \implies \Phi_x$ ha $2^q\cdot 2^{r(|x|)}$
        clausole soddisfatte da qualche assegnamento (da quello indotto da $w$).
        \item se $x \notin L$, $V$ accetta con probabilità $< \frac{1}{2}$ 
        qualunque sia $w$
    \end{enumerate}

    Quindi, si considera ora un certo problema $L$ facente parte di NP, e si sfrutta 
    la costruzione della formula appena descritta, ottenendo una certa $\Phi_x$, 
    questo è possibile perchè se $L$ è in NP esiste un verificatore PCP.
    
    Fissiamo $\bar{\epsilon} = \frac{1}{2^{q+1}}$.
    Supponiamo che esiste un algoritmo per MaxSat con tasso di approssimazione $(1+\bar{\epsilon})$, 
    sia $t^*(\Phi_x)$ la soluzione ottima dell'istanza di Sat.

    Distinguiuamo ora i due casi come prima:
    \begin{enumerate}
        \item se $x \in L \implies t^*(\Phi_x) = 2^q \cdot 2^{r(|x|)}$
        \item se $x \notin L \implies t^*(\Phi_x) \leq \frac{2^{r(|x|)}}{2} \cdot 2^q + \frac{2^{r(|x|)}}{2} 
        \cdot 2^{q-1}$, 
        ovvero, in metà dei casi risponde sì, mentre nell'altra metà sicuramente avrà un numero 
        di clausole risolte al più di $2^{q-1}$, visto che deve rispondere no, che 
        equivale a $\leq 2^{r(|x|)} \cdot 2^q - \frac{2^{r(|x|)}}{2}$
    \end{enumerate}

    Considerando l'algoritmo $(1+\bar{\epsilon})$-approssimante, nel primo caso, $x \in L$, individua una soluzione 
    che corrisponde a
    $$A = t(\Phi_x) \geq \frac{t^*(\Phi_x)}{1+\bar{\epsilon}} = \frac{2^q \cdot 2^{r(|x|)}}{\frac{1}{2^{q+1}}}$$
    Nel secondo caso invece, $x \notin L$
    $$B = t(\Phi_x) \leq t^*(\Phi_x) \leq 2^{r(|x|)} \cdot 2^q - \frac{2^{r(|x|)}}{2}$$
    Vale che $A < B$:
    \begin{equation}
        \begin{aligned}
            A - B = \frac{2^q \cdot 2^{r(|x|)}}{\frac{1}{2^{q+1}}}- 2^{r(|x|)} \cdot 2^q - \frac{2^{r(|x|)}}{2}\\
            = 2^{r(|x|)} \cdot \frac{2^{q+1} - 2^{q+1}(1+ \frac{1}{2^{q+1}}) + 1 + \frac{1}{2^{q+1}}}{2(1+ \frac{1}{2^{q+1}})} > 0
        \end{aligned}
    \end{equation}
    Questo significa che l'output emesso nel caso $x \in L$, è strettamente maggiore del caso in cui
    $x \notin L$, questo implica che, calcolando $A$, si può capire, osservando l'output dell'algoritmo, 
    a capire se $x \in L$. Ovvero in tempo polinomiale si è deciso $L$, assurdo.
\end{proof}

\begin{remark}
    La dimostrazione del lemma si può estendere a MaxE3Sat.
\end{remark}

\subsubsection{Inapprossimabilità Max Indipendent Set}
Il problema di Max Indipendent Set consiste in:

\emph{Input}: $G = (V,E)$\\
\emph{Output}: $X \subseteq V, \binom{X}{2}\cap E = \emptyset$\\
\emph{Costo}: $|X|$\\
\emph{Tipo}: max

\begin{theorem}
    Non è possibile approssimare in tempo polinomiale Max Indipendet Set a 
    meno di $(2-\epsilon)$ per $\epsilon > 0$.
\end{theorem}
\begin{proof}
    Si parte da un $L \subseteq 2^*, L \in NP$.

    Per il teorema PCP, esiste un verificatore $$V \in PCP[r(n), q], r(n) \in O(\log n), 
    q \in \mathbb{N}$$
    tale che $V$ accetta $L$.

    Fissato un $x \in 2^*$ simulo $V$ per ogni stringa $R \in 2^{r(|x|)}$ e per 
    ogni sequenza $\in 2^q$, per un totale di $2^{r(|x|)} \cdot 2^q$ simulazioni, 
    che sono in quantità polinomiale.

    Siano ora tutte le simulazioni che ritornano sì, così fatte:
    $$(R, \{i_1^R = v_1, \dots, i_q^R = v_q\})$$
    Queste coppie diventano nodi di un grafo, esiste un lato 
    tra due nodi, sse, dato 
    $$(R^\prime, \{i_1^R = v_1^\prime, \dots, i_q^R = v_q^\prime\})$$
    Non creo l'arco sse $R \neq R^\prime$ oppure se i secondi membri sono compatibili,
    sarebbero incompatibili se in una stessa posizione ci fossero due $v_i$ differenti.
    
    Per il grafo, vale che: 
    \begin{enumerate}
        \item Se $x \in L$, $G_x$ ha un insieme indipendente di cardinalità $\geq 2^{r(|x|)}$:
        questo perchè se $x \in L$ esiste un $w \in 2^q$ che accetta qualunque sia $R$. Ma allora 
        ci sono $2^{r(|x|)}$ coppie compatibili con $w$, visto che $w$ fa accettare tutti, 
        quindi non ci sono archi tra loro e di conseguenza vale l'osservazione.
        \item Se $x \notin L$, gli insiemi indipendenti di $G_x$ hanno cardinalità $\leq 2^{r(|x|) -1}$:
        supponendo che ne esista uno con cardinalità maggiore, esisterebbe un $w_s$ tale che 
        tutte le simulazioni rappresentate dall'insieme si verificherebbero. Questo non è corretto 
        poichè avremmo più della metà delle simulazioni verificate.
    \end{enumerate}

    Se ora esistesse un algoritmo $A$ con approssimazione $\alpha$, che 
    produce una soluzione di cardinalità $|S_{G_x}| = \frac{|S^*_{G_x}|}{\alpha}$
    \begin{enumerate}
        \item $x \in L \implies |S^*_{G_x}| \geq 2^{r(|x|)} \implies |S_{G_x}| \geq \frac{2^{r(|x|)}}{\alpha}$
        \item $x \notin L \implies |S^*_{G_x}| \leq \frac{2^{r(|x|)}}{2} \implies |S_{G_x}| \leq \frac{2^{r(|x|)}}{2}$
    \end{enumerate}
    Se $\alpha$ fosse $< 2$, i due casi sarebbero disgiunti, quindi deciderei in tempo polinomiale
    l'appartenenza di $x$ a $L$.
\end{proof}