\documentclass[a4paper]{article}

\usepackage[T1]{fontenc}
%\usepackage[utf8]{inputenc}
\usepackage[italian]{babel}
\usepackage{amssymb}
\usepackage{hyperref}
\usepackage{mathtools}
\usepackage{amsthm}
\usepackage[ruled,vlined,noend]{algorithm2e}

%%%%%%%%%%%%%%%%%%%%%%%%%%%%%%%%%%%%
\usepackage{listings} 
\lstdefinestyle{mystyle}{
    breakatwhitespace=false,                   
    captionpos=b,                    
    keepspaces=true,                 
    numbers=left,                    
    numbersep=5pt,                  
    showspaces=false,                
    showstringspaces=false,
    showtabs=False,                  
    tabsize=2
}

\lstset{style=mystyle}

\usepackage{setspace}
\singlespacing
%%%%%%%%%%%%%%%%%%%%%%%%%%%%%%%%%%%%

\mathtoolsset{showonlyrefs}  
\hypersetup{
    colorlinks=true,
    linkcolor=black,
    filecolor=black,      
    urlcolor=black,
}

\newcommand{\pluseq}{\mathrel{{+}{=}}}

\newtheorem {theorem}{Theorem}
\newtheorem{corollary}{Corollary}
\newtheorem{lemma}{Lemma}
\newtheorem{remark}{Remark}
\newtheorem{definition}{Definition}

\setcounter{secnumdepth}{3}
\setcounter{tocdepth}{3}

\title{Architectures for Big Data}
\author{Federico Bruzzone}
%\date{}
\makeindex

\begin{document}
\maketitle
\newpage
% \setlength{\parskip}{0.15em}
\tableofcontents
\setlength{\parindent}{0pt}
\setlength{\parskip}{0.8em}
\newpage

%\section{section}
...

\subsection{subsection}
	
\paragraph{paragraph}

\begin{remark}
    ...
\end{remark}

\subsubsection{subsubsection}

\subsubsection{subsubsection}

\paragraph{paragraph}

\subsection{subsection}
...
\begin{enumerate}
    \item 
    \item
    \item 
\end{enumerate}

\begin{itemize}
	\item 
    \item 
\end{itemize}

\begin{theorem}
    $PO \subseteq NPO$
\end{theorem}
\begin{proof}
    ...
\end{proof}

\begin{equation}
    \begin{aligned}
        \mathit{APX} = \{\Pi | \Pi \mathit{\;di\;ottimizzazione\;t.c.\;}
\exists \rho \geq 1, A, \\\mathit{t.c\;} x\rightarrow A \rightarrow y(x)\;\mathit{con}\;R_\Pi(x, y) \leq \rho\}
    \end{aligned}
\end{equation}


\paragraph{Algoritmo di risoluzione}
L'algoritmo di risoluzione è abbastanza semplice e si basa sulla 
ricerca di un cammino aumentante.

\begin{algorithm}[H]
    \SetAlgoLined
    \KwIn{$G=(V,E)$}
    \KwResult{Matching $M$ per $G$}
     $M \gets \emptyset$\\
     \While{$\Pi = \mathit{findAugmenting(G)}$}{
        $M.update(\Pi)$
     }
     \Return{M}
     \caption{BiMaxMatching}
\end{algorithm}


\subsection{Tecniche greedy}
Nella sezione a seguire si presentano problemi di ottimizzazione per cui 
tecniche greedy funzionano abbastanza bene. 
I problemi affrontati sono quelli di \emph{Load Balancing}, \emph{Center Selection} e \emph{Set Cover}.

\subsubsection{Load balancing}
\label{lb}
Il problema di Load Balancing può essere visto come il compito
di assegnare a macchine dei lavori da compiere, che richiedono del tempo, 
in modo da minimizzare il tempo totale.


\begin{theorem}
    Load Balancing è NPO completo
\end{theorem}
Bisogna perciò trovare un modo di approssimare una soluzione.
\paragraph{Greedy balance}
Il primo approccio alla risoluzione del problema è quello di assegnare la prossima
task alla macchina più scarica in questo momento.

\begin{algorithm}[H]
    \SetAlgoLined
    \KwIn{$M$ numero di macchine, $t_0, \dots, t_n$ task}
    \KwResult{Assegnamento delle task alle macchine}
     $L_i \gets 0\;\forall i \in M$\\
     $\alpha \gets \emptyset$\\
     \For{$j = 0, \dots, n$}{
         $\hat{i} = \min(L_i)$\\
         $\alpha(j) = \hat{i}$\\
         $L_{\hat{i}} \pluseq t_j$
     }
     \Return{$\alpha$}
     \caption{GreedyBalance}
\end{algorithm}

\begin{lstlisting} %[language=Python, caption=Python example]

import numpy as np
    
def incmatrix(genl1,genl2):
    m = len(genl1)
    n = len(genl2)
    M = None #to become the incidence matrix
    VT = np.zeros((n*m,1), int)  #dummy variable
    
    #compute the bitwise xor matrix
    M1 = bitxormatrix(genl1)
    M2 = np.triu(bitxormatrix(genl2),1) 

    for i in range(m-1):
        for j in range(i+1, m):
            [r,c] = np.where(M2 == M1[i,j])
            for k in range(len(r)):
                VT[(i)*n + r[k]] = 1;
                VT[(i)*n + c[k]] = 1;
                VT[(j)*n + r[k]] = 1;
                VT[(j)*n + c[k]] = 1;
                
                if M is None:
                    M = np.copy(VT)
                else:
                    M = np.concatenate((M, VT), 1)
                
                VT = np.zeros((n*m,1), int)
    
    return M
\end{lstlisting}

%%%%%%%%%%%%%%%%%
\usepackage{listings}
\usepackage{xcolor}

\definecolor{codegreen}{rgb}{0,0.6,0}
\definecolor{codegray}{rgb}{0.5,0.5,0.5}
\definecolor{codepurple}{rgb}{0.58,0,0.82}
\definecolor{backcolour}{rgb}{0.95,0.95,0.92}

\lstdefinestyle{mystyle}{
    backgroundcolor=\color{backcolour},   
    commentstyle=\color{codegreen},
    keywordstyle=\color{magenta},
    numberstyle=\tiny\color{codegray},
    stringstyle=\color{codepurple},
    basicstyle=\ttfamily\footnotesize,
    breakatwhitespace=false,         
    breaklines=true,                 
    captionpos=b,                    
    keepspaces=true,                 
    numbers=left,                    
    numbersep=5pt,                  
    showspaces=false,                
    showstringspaces=false,
    showtabs=false,                  
    tabsize=2
}

\lstset{style=mystyle}
\section{Course presentation}

%\subsection{}

%\subsubsection{}

The course aims at describing \textbf{big data processing framewokds}, both in terms of \textbf{methodologies} and \textbf{techonologies}.

Part of the lesson will focus on \textbf{Apache spark} and \textbf{distributed patterns}.

"May I ask..." a brave student voice break the presentation.

\textbf{It is not a spurious correlation}
\begin{itemize}
	\item What an Architecture is? 
	\item Why so I need to know this stuff?
	\item What is this "Hadoop"? Do I reallt need to know what a Name Node is?
	\item I would like to put a jBoss inside a Docker to allow Kubernetes load balancing it! (No! This is )
\end{itemize}

\subsubsection{How to use Python}	
\textbf{We are condidering Python 3+}
\begin{itemize}
	\item version > 3 is incompatible with previus version;
	\item version 2.7 is the current version.
\end{itemize}
\textbf{A python program can be:}
\begin{itemize}
	\item edited in the python shell and executed step-by-step by the shell;
	\item edited and run through the iterpreter.
\end{itemize}

\subsection{Overview of the Basic Concepts}	

\subsubsection{Our first Python program}
\hrule
\begin{lstlisting}[language=Python, caption=humanize.py]
SUFFIXES = {1000: ['KB', 'MB', 'GB', 'TB', 'PB', 'EB', 'ZB', 'YB'], 
            1024: ['KiB', 'MiB', 'GiB', 'TiB', 'PiB', 'EiB', 'ZiB', 'YiB']}
def approximate_size(size, a_kilobyte_is_1024_bytes=True):
	''' Convert a file size to human-readable form. '''
	if size < 0:
		raise ValueError('number must be non-negative')
	multiple = 1024 if a_kilobyte_is_1024_bytes else 1000
	for suffix in SUFFIX[multiple]:
		size /= multiple
		if size < multiple:
			return '{0:.1f} {1}'.format(size, suffix)
		raise ValueError('number too large')
				
if __name__ == '__main__':
	print(approximate_size(1000000000000, False))
	print(approximate_size(1000000000000))
\end{lstlisting}
\hrule	

\subsubsection{Declaring function}	
\textbf{Python has function}
\begin{itemize}
	\item no header files à la C/C++;
	\item no interface/implementation à la Java.
\end{itemize}			
\hrule
\begin{lstlisting}[language=Python]
def approximate_size(size, a_kilobyte_is_1024_bytes=True):
\end{lstlisting}	
\begin{enumerate}
	\item \textbf{def}: function definition keyword;
	\item \textbf{approximate\_size}: function name;
	\item \textbf{a\_kilobyte\_is\_1024\_bytes}: comma separate argument list;
	\item \textbf{=True}: default value.
\end{enumerate}
\hrule		
\textbf{Python has function}
\begin{itemize}
	\item no return type, it always return a value (\textbf{None} as a default); 
	\item no parameter types, the interpreter figures out the parameter type.
\end{itemize}	

\subsubsection{Calling Functions}
\textbf{Look at the bottom of the \textit{humanize.py} program}
\hrule
\begin{lstlisting}[language=Python]
if __name__ == '__main__':
	print(approximate_size(1000000000000, False))
	print(approximate_size(1000000000000))
\end{lstlisting}
\begin{enumerate}
	\item[2] in this call to \textbf{approximate\_size()}, the \textbf{a\_kilobyte\_is\_1024\_bytes} parameter will be \textbf{False} since you explicitly pass it to the function;
	\item[3] in this row we call  \textbf{approximate\_size()} with only a value, the parameter \textbf{a\_kilobyte\_is\_1024\_bytes} will be \textbf{True} as defined in the function declaration.
\end{enumerate}
\hrule
\textbf{Value can be passed by name as in}:
\hrule
\begin{lstlisting}[language=Python]
def approximate_size(a_kilobyte_is_1024_bytes=True, size=1000000000000)
\end{lstlisting}	
\hrule
\textbf{Parameters' order is not relevant}

\subsubsection{Writing readable code}
\textbf{Documentation Strings}
A python function can be documented by a documentation string (docstring for short).
\begin{center}
	\textit{''' Convert a file size to human-readable form.  '''}
\end{center}
\textbf{Triple quotes delimit a single multi-string}
\begin{itemize}
	\item if it immediatly follows the function's declaration it is the doc-string associated to the function;
	\item docstrings can be retrieved at run-time (they are attributes).
\end{itemize}
\textbf{Case-Sensitive}
All names in Python are case-sensitive

\subsubsection{Everything is an object}
\textbf{Everything in Python is an object, functions included}
\begin{itemize}
	\item \textbf{import} can be used to load python programs in the system as modules;
	\item the dot-notation gives access to the the public functionality of the imported modules;
	\item the dot-notation can be used to access the attributes (e.g., the \textbf{\_\_doc\_\_})
	\item \textbf{humanize\.approximate\_size.\_\_doc\_\_} gives access to the docstring of the \textbf{approximate\_size()} function; the docstring is stored as an attribute.
\end{itemize}

\subsubsection{Everything is an object (Cont'd)}
\textbf{In python is an object, better, is a \textbf{first-calss object}}
\begin{itemize}
	\item everything can be assigned to a variable or passed as an argument
\end{itemize}
\hrule
\begin{lstlisting}[language=Python]
h1 = humanize.approximate_size(9128)
h2 = humanize.approximate_size
\end{lstlisting}	
\begin{itemize}
	\item \textbf{h1} contains the string calculated by \textbf{approximate\_size(9128};
	\item \textbf{h2} contains the "function" object \textbf{approximate\_size()}, the result is not calculated yet;
	\item to simplify the concept: \textbf{h2} can be considered as a new name of (alias to) \textbf{approximate\_size}.
\end{itemize}
\hrule

\subsubsection{Indenting code}
\textbf{No explicit block delimiters}
\begin{itemize}
	\item the only delimiter is a column (':') and the code indentation;
	\item code blocks (e.g., functions, if statements, loops, ...) are defined by their indentation;
	\item white spaces and tabs are relevant: use them consistently;
	\item indentation is checked by the compiler.
\end{itemize}

\subsubsection{Exceptions}
\textbf{Exceptions are Anomaly Situations}
\begin{itemize}
	\item C encourages the use of return codes which you check;
	\item Python encourages the use of exceptions which you handles.
\end{itemize}
\textbf{Raising Exceptions}
\begin{itemize}
	\item the \textbf{raise} statement is used to rise an exception as in:
\begin{lstlisting}[language=Python]
raise ValueError('number must be non-negative')
\end{lstlisting}
	\item syntax recalls function calls: \textbf{raise} statement followed by an exception name with an optional argument;
	\item exceptions are relized by classes.
\end{itemize}
\textbf{No need to list the exceptions in the function declaration handling Exceptions}
\begin{itemize}
	\item an exception is handled by a \textbf{try} ... \textbf{except} block.
\hrule
\begin{lstlisting}[language=Python]
try:
	from lxml import etree
except ImportError:
	import xml.etree.ElementTree as etree
\end{lstlisting}
\hrule
\end{itemize}

\subsubsection{Running scripts}
\textbf{Look again, at the bottom of the \textit{humanize.py} program}:
\hrule
\begin{lstlisting}
if __name__ == '__main__':
	print(approximate_size(1000000000000, False))
	print(approximate_size(1000000000000))
\end{lstlisting}
\hrule
\textbf{Modules are Objects}
\begin{itemize}
	\item they have a built-in attribute \textbf{\_\_name\_\_}
\end{itemize}
\textbf{The value of \textbf{\_\_name\_\_} depends on how you call it}
\begin{itemize}
	\item if imported it contains the name of the file without path and extension.
\end{itemize}



\end{document}