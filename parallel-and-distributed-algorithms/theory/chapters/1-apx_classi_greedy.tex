\section{Algoritmi di approssimazione}
In questa parte si introdurranno gli algoritmi di approssimazione, 
dopo aver accennato ad alcuni concetti preliminari legati alla complessità.
Si definiranno in particolare classi di problemi e alcune 
tecniche note nella risoluzione di problemi di ottimizzazione.

\subsection{Notazioni}

\paragraph{Insiemi numerici come interi}
Si indicheranno alcuni insiemi numerici con un singolo intero, 
es $$k = \{0, 1, \dots, k-1\}$$

\paragraph{Insieme delle funzioni} L'insieme delle funzioni che hanno come dominio $A$ e codominio $B$ si indica come segue: 
$$B^A = \{ f | f : A \rightarrow B\}$$

\paragraph{Insieme delle parti} L'insieme dei sottoinsiemi di $A$ si indica con $2^A = P(A)$.

\paragraph{Alfabeto} Fissato un alfabeto $\Sigma$, indichiamo con $\Sigma^*$ l'insieme delle stringhe ottenute da 
$\Sigma$.

\paragraph{Esempio}
La notazione $2^{2^*}$ indica l'insieme dei linguaggi binari.

\subsection{Classi di complessità}
Partiamo dalla definizione di algoritmo, per poi arrivare a definire la 
complessità algoritmica e strutturale.
\paragraph{Algoritmo}
Un algoritmo per un problema $\Pi$ è una macchina di Turing, che verrà indicata con $A$, che opera come segue:
$$x \in I_{\Pi} \longrightarrow A \longrightarrow y \in O_{\Pi}\;\;t.c.\;\;y \in Sol_{\Pi}(x)$$
$$Sol_{\Pi}: I_\Pi \longrightarrow (2^{O_\Pi} \setminus \{\emptyset\})$$
La funzione associa ad un input l'insieme dei suoi output corretti.

\begin{remark}
    Gli input di un problema devono essere codificati in qualche modo. 
    Ad esempio, considerando una macchina di Turing, gli input sono 
    pensati come stringhe binarie. 
    Esistono molti modi di codificare per esempio due interi, si potrebbe pensare di utilizzare la coppia di Cantor, oppure di raddoppiare i bit.
    La complessità di un algoritmo non varia se le codifiche sono 
    polinomialmente contenute le une nelle altre. Ovvero, una codifica esponenziale cambierebbe la complessità rispetto a una codifica polinomiale, 
    ma nel nostro caso non succederà.
\end{remark}

\subsubsection{Complessità algoritmica}
La complessità algoritmica è lo studio del dispendio di risorse di 
un algoritmo. Sostanzialmente ci si sta chiedendo come un algoritmo risolve un certo problema.

Una delle risorse considerabili è il tempo: $$T_A : I_\Pi \rightarrow \mathbb{N}$$
Alternativamente possiamo considerare questa funzione, che ragione per taglia piuttosto 
che per singolo input: 
$$t_A : \mathbb{N} \rightarrow \mathbb{N}$$ dove il dominio è 
la lunghezza di input, in quel caso si avrà che 
$$t_A(n) : \max\{ T_A(x),\; x \in I_{\Pi},\; |x| = n\}$$

Si possono considerare limiti alla complessità algoritmica, un upper bound si ottiene 
esibendo una soluzione con quella complessità, per esempio, trovo una soluzione quadratica 
al problema dell'ordinamento. Posso anche trovare un lower bound, per esempio, 
dimostro che ordinare $n$ valori richiede almeno $O(n \log n)$ operazioni.

\subsubsection{Complessità strutturale}
Si definiscono ora due classi di problemi, \emph{P} e \emph{NP}, per 
poi introdurre i concetti di \emph{riducibilità polinomiale} e di \emph{NP-completezza}.

\paragraph{Classe P}
La classe \emph{P} corrisponde ai problemi decisionali 
per cui esiste un algoritmo che opera
in tempo polinomiale, ovvero: 
$$P = \{\Pi\;|\;\Pi\;decisionale, \; \exists\;A\;per\;\Pi,\;t.c.\;t_a(n)\;=\;O(Polinomio)\}$$

\paragraph{Classe NP}
La classe \emph{NP} corrisponde ai problemi decisionali 
per cui esiste un algoritmo che opera
in tempo polinomiale su una macchina non deterministica, ovvero: 
\begin{equation}
    \begin{aligned}
        \mathit{NP} = \{\Pi\;|\;\Pi\;decisionale, \; \exists\;A\;per\;\Pi,\; t.c.\;t_a(n)\;=\;O(Polinomio)\;
        \\su\;una\;macchina\;non\;deterministica\}
    \end{aligned}
\end{equation}

\begin{remark}
    Un problema è decisionale se dato un input da un output binario.
\end{remark}

\paragraph{Riducibilità polinomiale}
Un problema $\Pi_1$ si dice riducibile polinomialmente ad un problema $\Pi_2$ se esiste 
un mapping tra input del primo e del secondo problema che rispetta le seguenti proprietà:
$$\Pi_1 \leqslant _p \Pi_2 \;sse\; \exists f : 2^* \rightarrow 2^*\; t.c$$
\begin{enumerate}
    \item $f$ è calcolabile in tempo polinomiale
    \item $\forall x \in I_{\Pi1},\;f(x) \in I_{\Pi2},\; Sol_{\Pi1}(x) = Sol_{\Pi2}(f(x))$ 
\end{enumerate}

\paragraph{NP completezza}
Un problema $\Pi$ è NP completo sse $\forall\;\Pi^\prime \in NP,\; \Pi^\prime \leqslant _p \Pi, \; \Pi \in NP$.
Ovvero se ogni problema in NP è riducibile polinomialmente al problema che si sta 
considerando.

\begin{theorem}
    SAT è NP completo.
\end{theorem}
\begin{corollary}
    Se $\Pi_1 \leqslant _p \Pi_2$ e $\Pi_1$ è NP completo, allora $\Pi_2$ è NP completo.
\end{corollary}
\begin{proof}
    $\Pi^\prime \in NP, \Pi^\prime \leqslant _p \Pi_1 \leqslant _p \Pi_2$, quindi $\Pi_2$ è NP completo.
\end{proof}
\begin{remark}
    Se trovassi un problema $\Pi$ NP completo t.c, $\Pi^\prime \leqslant _p \Pi$, 
    allora \emph{P=NP}.
\end{remark}

\subsection{Problemi di ottimizzazione}
Nella definizione di un problema di ottimizzazione bisogna tenere conto dei 
seguenti parametri:
\begin{enumerate}
    \item Insieme di input $I_\Pi \subseteq 2^*$;
    \item Insieme di output $O_\Pi \subseteq 2^*$;
    \item $F_\Pi : I_\Pi \rightarrow 2^{O_\Pi} \setminus \{\emptyset\}$, $F_\Pi(x)$ indica 
    le soluzioni accettabili per l'input $x$;
    \item $C_\Pi : I_\Pi \times O_\Pi \rightarrow \mathbb{Q}^{>0}$, funzione obiettivo, 
    con $C_\Pi(x,y)$ si indica il valore della funzione obiettivo per l'input \emph{x}, con soluzione $y \in O_\Pi$;
    \item $t_\Pi \in \{min, \;max\}$, ovvero il criterio del problema.
\end{enumerate}
L'insieme delle soluzioni per $x$ si può definire come:
$$Sol(x) = \{y^* \in O_\Pi | y^* \in F_\Pi, \forall y^\prime \in F_\Pi, c(x, y^*) \leqslant (\geq ) c(x, y^\prime)\}$$
Ovvero l'insieme delle $y$ possibili per quella fissata $x$, tale che minimizzano o massimizzano la funzione $C_\Pi$.

\paragraph{Problema di decisione associato}
Dato un problema di ottimizzazione $\Pi$ esiste un problema di decisione associato $\hat{\Pi}$.
\\L'idea è quella di considerare un input del problema originale e un costo alla soluzione, e rispondere
in base all'esistenza di una soluzione con quel costo.\\
In particolare: $$I_{\hat{\Pi}} = I_\Pi \times \mathbb{Q}^{>0}$$ 
$$(x, \theta) = C_\Pi(x, y^*(x)) \leqslant (\geq ) \theta$$

Ovvero, rispondo sì se e solo se il costo della soluzione ottima è minore, 
o maggiore, di $\theta$.
Pensando al problema MaxSat, che massimizza il numero di clausole risolte. 
Posso considerare il problema di decisione associato, che preso in input 
una CNF e un intero, risponde si, se esiste un assegnamento che soddisfa quel
numero di clausole.

Se avessi un algoritmo per il problema di ottimizzazione, avrei anche quello 
di decisione, non vale il contrario, perché dovrei iterare tutti gli interi
per capire il bound del problema di ottimizzazione.

\subsubsection{Classi di ottimizzazione}
Data una definizione di problema di ottimizzazione e individuato il problema di decisione 
associato, si passa ora alla definizione delle classi di problemi di ottimizzazione, \emph{PO}
e \emph{NPO}.

\paragraph{PO}
La classe PO equivale, nel mondo dei problemi di ottimizzazione, alla classe P,
più formalmente:
$$PO = \{ \Pi\;|\;\Pi\;di\;ottimizzazione\;e\;polinomiale\}$$

\paragraph{NPO}
Fanno parte della classe NPO quei problemi di ottimizzazione non risolvibili 
in tempo polinomiale, o meglio:
\begin{enumerate}
    \item $I_\Pi, O_\Pi \in P$
    \item Esiste un polinomio \emph{Q} tale che:
        \begin{itemize}
            \item $\forall x \in I_\Pi, \forall y \in F_\Pi, |y| \leq Q(|x|)$
            \item $\forall x \in I_\Pi, \forall y \in 2^*, se |y| \leq Q(|x|)$ decidere
            se $t \in F_\Pi$ è polinomiale
        \end{itemize}
    \item $c_\pi$ è calcolabile in tempo polinomiale
\end{enumerate}
L'idea consiste poi nel generare tutte le soluzioni ammissibili, su una macchina 
non deterministica.
La valutazione delle soluzioni avviene poi in tempo polinomiale, per i vincoli espressi
sopra.

\begin{theorem}
    $PO \subseteq NPO$
\end{theorem}
\begin{theorem}
    Se $\Pi \in \mathit{PO}, \hat{\Pi} \in P$, 
    se $\Pi \in \mathit{NPO}, \hat{\Pi} \in \mathit{NP}$ 
\end{theorem}
\begin{proof}
    Dato un problema $\Pi \in P, \exists A$ polinomiale, tale che
    $$x \in I_\Pi \longrightarrow A \longrightarrow y^*(x)$$
    Dato il problema di decisione associato, esiste $\hat{A}$ tale che:
    $$(x, \theta) \in I_\Pi \times \mathbb{Q}^{>0} \longrightarrow \hat{A} \longrightarrow yes/no$$
    L'output sarà yes sse $c^*(x) \geq (\leq) \theta$, con $c^*(x) = c_\Pi(x, y^*(x))$.\\
    
    L'algoritmo $\hat{A}$ sarà polinomiale, infatti, basta applicare \emph{A} ad \emph{x} 
    per individuare $y^*(x)$ per poi confrontarlo con $\theta$. 

    Similmente, nel caso in cui il problema appartenga a \emph{NP}, sia $A$ che $\hat{A}$ saranno non
    deterministiche.
\end{proof}

\paragraph{NPO completezza}
Un problema si dice NPO completo sse il problema di decisione associato è NP completo.\\
$$\mathit{NPOcomp} = \{\Pi \in \mathit{NPO} | \hat{\Pi} \in \mathit{NP}\}$$

\begin{theorem}
    Se $\Pi \in \mathit{NPOcomp}$, $\Pi \notin PO$, a meno che $P=NP$
\end{theorem}
\begin{proof}
    Supponiamo di avere $\Pi \in \mathit{NPOcomp}$ tale che $\Pi \in PO$.\\
    Allora avremmo che per la prima affermazione $\hat{\Pi} \in \mathit{NPcomp}$ e per la seconda, 
    $\hat{\Pi} \in P$, assurdo.
\end{proof}

\paragraph{Rapporto di approssimazione}
Dato $\Pi$ problema di ottimizzazione e siano $x \in I_\Pi, y \in F_\Pi(x)$ definiamo
rapporto di approssimazione:
$$R_\Pi(x, y) = \max\bigg\{\frac{c(x, y)}{c^*(x)}, \frac{c^*(x)}{c(x, y)}\bigg\}\geq 1$$
In un problema di massimo dominerà il secondo termine, mentre in uno di minimo il primo, 
si può perciò esprimere il rapporto senza dipendere dal tipo di problema.

\paragraph{APX}
I problemi in APX hanno rapporto di approssimazione limitato da una certa quantità, 
formalmente: 
\begin{equation}
    \begin{aligned}
        \mathit{APX} = \{\Pi | \Pi \mathit{\;di\;ottimizzazione\;t.c.\;}
\exists \rho \geq 1, A, \\\mathit{t.c\;} x\rightarrow A \rightarrow y(x)\;\mathit{con}\;R_\Pi(x, y) \leq \rho\}
    \end{aligned}
\end{equation}
La definizione di classe APX permette una stratificazione per $\rho$, vale che APX 
con valore 1 equivale alla classe PO.

\paragraph{PTAS}
I problemi in PTAS hanno rapporto di approssimazione limitato da una certa quantità,
che si può decidere in modo arbitrario, formalmente: 
\begin{equation}
    \begin{aligned}
        \mathit{PTAS} = \{\Pi | \Pi \mathit{\;di\;ottimizzazione\;t.c.\;}\exists A, \\
\mathit{t.c\;} (x, r) \in I_\Pi \times \mathbb{Q}^{>1} \rightarrow A \rightarrow y \\
y \in F_\Pi, R_\Pi(x, y) \leq r, \\\mathit{polinomiale\;in\;} |x|\}
    \end{aligned}
\end{equation}
Si fa notare che $r$ può essere molto vicino ad 1, ma il tempo polinomiale
potrebbe aumentare esponenzialmente.

\subsubsection{BiMaxMatching}
Si presenta ora un problema della classe PO, ovvero un problema di ottimizzazione
per cui esiste un algoritmo esatto che opera in tempo polinomiale.

\emph{Input}: grafo non orientato $G = (V,E)$\\
\emph{Output}: matching $M \subseteq E$ tale che $\forall x\;\exists!\;xy \in M$\\
\emph{Costo}: $|M|$\\
\emph{Tipo}: $\max$

A seguire l'algoritmo esatto polinomiale per i grafi bipartiti, esiste anche 
per grafi generici.

\paragraph{Cammino aumentante per M}
Nel grafo in cui si sta cercando il matching esistono due tipi di vertici:
\begin{enumerate}
    \item Occupati: incide un lato di $M$
    \item Liberi: non usati nel matching
\end{enumerate}
Un cammino aumentante è un cammino semplice che:
\begin{enumerate}
    \item Parte e arriva su un vertice libero
    \item Alterna lati che appartengono a $M$ e ad $E\setminus M$
\end{enumerate}

\begin{theorem}
    Se $\exists$ un cammino aumentante per $M$, $M$ non è massimo.
\end{theorem}
\begin{proof}
    Individuato un cammino aumentante, ci saranno lati che appartengono al matching, $X$ 
    e lati che non ci appartengono, $Y$. Questi ultimi sono in quantità maggiore per la definizione
    di cammino aumentante. Posso quindi rimuovere da $M$ i lati in $X$ e aggiungere quelli in $Y$.
\end{proof}
\begin{theorem}
    Se M non è massimo, esiste un cammino aumentante.
\end{theorem}
\begin{proof}
    Se $M$ non è massimo significa che $\exists M^\prime, |M^\prime| \geq |M|$.\\
    Sia $X = M^\prime \Delta M = (M^\prime \setminus M) \cup (M \setminus M^\prime)$

    Si osserva che, su ogni vertice incidono al massimo due lati di $X$, poichè 
    nei due matching c'è al massimo un lato che indice su ogni vertice.

    Rappresentando graficamente $X$ si noterebbero solo cammini cicli o nodi singoli, visto che
    ogni nodo ha grado 0,1 oppure 2.

    Considerando un ciclo in $X$ si nota che: 
    \begin{itemize}
        \item due lati consecutivi fanno parte di matching 
        differenti, altrimenti avrei più lati incidenti su uno stesso vertice per matching
        \item i cicli hanno lunghezza pari
    \end{itemize}
    Si nota poi che: $$|M^\prime| \geq |M| \implies (M^\prime \setminus M) \geq (M \setminus M^\prime)$$ 
    quindi, dato che i cicli in $X$ hanno in egual quantità lati di $M^\prime$ e $M$, deve 
    esitere un cammino. Tale cammino avrà per forza più lati in $M^\prime$ che $M$, inoltre
    i lati si alternano tra i due matching e iniziano e finiscono con 
    lati che non appartengono a $M$, quindi è un cammino aumentante per $M$.
\end{proof}
\begin{theorem}
    Siano $G = (V,E)$ un grafo bipartito e sia $M \subseteq E$ un suo matching, sono 
    equivalenti:
    \begin{enumerate}
        \item M è massimo
        \item Non esiste un cammino aumentante per M
    \end{enumerate}
\end{theorem}

\paragraph{Algoritmo di risoluzione}
L'algoritmo di risoluzione è abbastanza semplice e si basa sulla 
ricerca di un cammino aumentante.

\begin{algorithm}[H]
    \SetAlgoLined
    \KwIn{$G=(V,E)$}
    \KwResult{Matching $M$ per $G$}
     $M \gets \emptyset$\\
     \While{$\Pi = \mathit{findAugmenting(G)}$}{
        $M.update(\Pi)$
     }
     \Return{M}
     \caption{BiMaxMatching}
\end{algorithm}
Siano $v_1$ e $v_2$ le parti destra e sinistra rispettivamente di un grafo bipartito.\\
Un'idea per sviluppare la funzione di cammino aumentante su grafi bipartiti è quella
di considerare solo archi che non appartengono a $M$ mentre si va da $v_1$ a $v_2$ e 
solo quelli che appartengono a $M$ nella direzione opposta.
Se trovo un cammino del genere aggiorno $M$.
\begin{remark}
    Il problema può anche essere risolto con una rete di flusso, aggiungendo un nodo
    sorgente collegato alla partizione sinistra dei nodi e un nodo pozzo alla partizione
    destra. Si applica poi uno dei tanti algoritmi per il flusso.
\end{remark}
\begin{remark}
    Una variante del problema è quella del perfect matching, che capita quando 
    la cardinalità di $v_1$ e $v_2$ equivale a $\frac{n}{2}$. Successivamente si 
    verifica se $M = \mathit{BiMaxMatching}(G)$ ha cardinalità $\frac{n}{2}$.
\end{remark}

\subsection{Tecniche greedy}
Nella sezione a seguire si presentano problemi di ottimizzazione per cui 
tecniche greedy funzionano abbastanza bene. 
I problemi affrontati sono quelli di \emph{Load Balancing}, \emph{Center Selection} e \emph{Set Cover}.

\subsubsection{Load balancing}
\label{lb}
Il problema di Load Balancing può essere visto come il compito
di assegnare a macchine dei lavori da compiere, che richiedono del tempo, 
in modo da minimizzare il tempo totale.

\emph{Input}: $t_0, \dots, t_n \in \mathbb{N}^{>0}$ task, $m \in \mathbb{N}^{>0}$ macchine\\ 
\emph{Output}: $\alpha : n \rightarrow m$ \\
\emph{Costo}: $L = \max_{i \in n}(L_i)$, $L_i = \sum_{j \in \alpha^{-1}(i)}t_j$ \\
\emph{Tipo}: $\min$

\begin{theorem}
    Load Balancing è NPO completo
\end{theorem}
Bisogna perciò trovare un modo di approssimare una soluzione.
\paragraph{Greedy balance}
Il primo approccio alla risoluzione del problema è quello di assegnare la prossima
task alla macchina più scarica in questo momento.

\begin{algorithm}[H]
    \SetAlgoLined
    \KwIn{$M$ numero di macchine, $t_0, \dots, t_n$ task}
    \KwResult{Assegnamento delle task alle macchine}
     $L_i \gets 0\;\forall i \in M$\\
     $\alpha \gets \emptyset$\\
     \For{$j = 0, \dots, n$}{
         $\hat{i} = \min(L_i)$\\
         $\alpha(j) = \hat{i}$\\
         $L_{\hat{i}} \pluseq t_j$
     }
     \Return{$\alpha$}
     \caption{GreedyBalance}
\end{algorithm}
Il tempo è polinomiale, si ottiene una complessità $O(nm)$, implementando 
la ricerca del minimo con coda di priorità si ottiene $O(n\log(m))$.
\begin{theorem}
    Greedy balance è 2-approssimante per load balancing, ovvero la soluzione
    individuata è al massimo il doppio di quella ottima.
\end{theorem}
\begin{proof}
    Sia $L^*$ il costo della soluzione ottima, il suo costo non è minore
    della somma delle task diviso il numero di macchine:
    $$L^* \geq \frac{1}{m}\sum_{i \in n} t_i$$
    Infatti, sommando i carichi, si ottiene la somma dei task:
    $$\sum_{i \in m} L_i^* = \sum_{i \in n} t_i$$
    $$\frac{1}{m}\sum_{i \in m} L_i^* = \frac{1}{m}\sum_{i \in n} t_i 
    \implies \exists L_i^* \geq \frac{1}{m}\sum_{i \in n} t_i 
    \implies L^* \geq \frac{1}{m}\sum_{i \in n} t_i$$
    
    Si osserva poi che:
    $$L^* \geq \max_{i \in n}(t_i)$$

    Sia ora $L$ la soluzione individuata da Greedy Balance, e $\hat{i}$
    la macchina con carico massimo $L_{\hat{i}} = L$.\\
    Sia $\hat{j}$ l'ultimo task assegnato a $\hat{i}$. Il carico 
    che la macchina aveva prima dell'ultimo task era:
    $$L_{\hat{i}}^\prime = L_{\hat{i}} - t_{\hat{j}}$$
    La scelta greedy dell'algoritmo implica che:
        $$L_{\hat{i}} - t_{\hat{j}} \leq L_{i}^\prime \forall i \implies L_{\hat{i}} - t_{\hat{j}} \leq L_{i} \forall i$$
    Dove $L_i^\prime$ indica il carico della macchina i prima di assegnare $\hat{j}$. 
    Ottengo $m$ disequazioni che posso sommare tra loro:
    \begin{equation}
        \begin{aligned}
            m(L_{\hat{i}} - t_{\hat{j}}) &\leq \sum_{i \in m}L_{i} =  \sum_{i \in n}t_{i} && \text{divido per m}\\
            L_{\hat{i}} - t_{\hat{j}} &\leq \frac{1}{m}\sum_{i \in n}t_{i} \leq L^* && \text{dall'osservazione iniziale}\\
            L &= L_{\hat{i}} = (L_{\hat{i}} - t_{\hat{j}}) + t_{\hat{j}}  && \text{uso la riga sopra e la seconda osservazione}\\
            L &\leq L^* + L^* \leq 2L^*
        \end{aligned}
    \end{equation}

\end{proof}
\begin{corollary}
    Load balancing appartiene a APX    
\end{corollary}
\begin{theorem}
    $\forall \epsilon > 0$ esiste un input su cui Greedy Balance produce una soluzione 
    $L$ t.c. $$L-\epsilon \leq \frac{L}{L^*}\leq2$$
\end{theorem}
\begin{proof}
    Si considerano $m > \frac{1}{\epsilon}$ macchine e $n = m(m-1) + 1$ task.\\
    Tutti i task ad eccezione dell'ultimo hanno lunghezza 1, mentre l'ultimo ha lunghezza $m$.

    L'assegnamento di Greedy Balance assegna tutte le task grandi 1 a tutte le macchine, 
    infine si assegna l'ultima task lunga $m$ a una macchina causale, visto che sono 
    tutte piene allo stesso modo. Il tempo totale è $L = 2m-1$.

    Esiste però un assegnamento migliore, ovvero, alla prima macchina si assegna la task lunga $m$, 
    mentre alle altre tutte quelle lunghe 1 in modo uniforme. In questo caso si ottiene
    $L^* = m$.
    $$\frac{L}{L^*} = \frac{2m-1}{2} = 2 - \frac{1}{m} \geq 2-\epsilon$$
\end{proof}

\paragraph{Sorted Balance}
L'algoritmo dapprima ordina i task in ordine inverso di lunghezza e poi effettua la scelta 
greedy vista in precendeza.

\begin{algorithm}[H]
    \SetAlgoLined
    \KwIn{$M$ numero di macchine, $t_0, \dots, t_n$ task}
    \KwResult{Assegnamento delle task alle macchine}
    $tasks \gets rev(sorted(t_i, \dots t_n))$\\
    \Return{$GreedyBalance(M, tasks)$}
     \caption{SortedBalance}
\end{algorithm}

\begin{theorem}
    Sorted Balance fornisce una $\frac{3}{2}\textit{-approssimazione}$  a Load Balancing
\end{theorem}
\begin{proof}
    Se $N\leq M$ l'algoritmo trova la soluzione ottima, si considera quindi il caso di 
    avere più task che macchine.
    
    Si osserva che: $$L^* \geq 2t_m$$ questo vale poichè dei primi $m+1$ task
    almeno $2$ sono assegnati ad una stessa macchina (principio delle camicie e dei 
    cassetti) e quei due task sono $\geq t_m$ (task ordinati in ordine decrescente).
    
    Sia poi $\hat{i}$ la macchina tale che $L_{\hat{i}} = L$ e sia 
    $\hat{j}$ l'ultimo task assegnato alla macchina $\hat{i}$\footnote{La macchina ha almeno $2$ task, 
    altrimenti abbiamo trovato la soluzione ottima}.
    Si osserva che: 
    \begin{equation}
        \begin{aligned}        
            \hat{j} &\geq m\\
            t_{\hat{j}} &\leq t_m  \leq \frac{1}{2}L^* && \text{dall'osservazione iniziale}\\
            L &= L_{\hat{i}} = (L_{\hat{i}} - t_{\hat{j}}) + t_{\hat{j}} && \text{vale la dimostrazione precedente}\\
            L &\leq L^* + \frac{1}{2}L^* \leq \frac{3}{2}L^*
        \end{aligned}
    \end{equation}
\end{proof}
\begin{remark}
    In realtà Sorted Balance è $\frac{4}{3}\textit{-approssimante}$
\end{remark}
\begin{remark}
    Load Balancing $\in$ PTAS
\end{remark}

\subsubsection{Center selection}
Il problema della selezione dei centri può essere visto come il compito 
di eleggere alcune città come centri in un insieme di città. 
\\L'obiettivo è quello di minimizzare la distanza della città più sfortunata
dal proprio centro. Si sta lavorando in uno spazio metrico.

\emph{Input}: $S \subseteq \Omega$ dove $(\Omega, d)$ è uno spazio metrico, $K$ centri da selezionare\\
\emph{Output}: $C \subseteq S$, $|C| \leq K$, $C : S \longrightarrow C$ funzione che manda 
un punto in S nel centro che minimizza la distanza da esso\\
\emph{Costo}: $\rho(C) = \max_{s \in S}d(s,C(s))$ \\
\emph{Tipo}: $\min$

\begin{definition}

    Uno spazio si dice metrico se esiste il concetto di distanza in esso.
    Una funzione di distanza è: 
    $$d : \Omega \times \Omega \longrightarrow \mathbb{R} $$
    E rispetta le seguenti proprietà
    \begin{itemize}
        \item $d(x,y) \geq 0 $, $d(x,y) = 0 \implies x = y$
        \item $d(x,y) = d(y,x)$
        \item $d(x,y) \leq d(x,z) + d(z,y)$
    \end{itemize}
\end{definition}

\paragraph{Center selection plus}
L'algoritmo che segue ha un input aggiuntivo $r$, idealmente dovrebbe equivalere a 
$\rho^*$, ovvero la soluzione ottima.

\begin{algorithm}[H]
    \SetAlgoLined
    \KwIn{$S \subseteq \Omega$, $K \in \mathbb{N}^{>0}$, $r \in \mathbb{R}^{>0}$}
    \KwResult{Selezione dei centri}
    $C \gets \emptyset$\\
    \While{$S \neq \emptyset$}{
        $\bar{s} = \mathit{random}(S)$\\
        $C.add(\bar{s})$\\
        \ForAll{$e \in S$}{
            \If{$d(e, \bar{s}) \leq 2r$}{
                $S.pop(e)$
            }
        }
    }
    \Return{$(|C| \leq K)?\;C\;:\;\mathit{impossible}$}
     \caption{CenterSelectionPlus}
\end{algorithm}


\begin{theorem}
    Considerando l'algoritmo proposto, valgono:
    \begin{enumerate}
        \item se l'algoritmo emette un $C$, l'approssimazione è $\leq \frac{2r}{\rho^*}$
        \item se $r \geq \rho^*$, l'algoritmo emette un output
        \item se il risultato è impossibile, allora $r < \rho^*$
    \end{enumerate}
\end{theorem}
\begin{proof}
    Partendo dal punto 1, si osserva che: $\forall s \in S$, la cancellazione
    di $s$ implica che la sua distanza da uno dei centri fosse $\leq 2r$, segue che, 
    ogni punto è al massimo distante $2r$ da un centro.
    \begin{equation}
        \begin{aligned}
            \rho(C) \leq 2r && \text{per il discorso precedente}\\
            \frac{\rho(C)}{\rho^*(C)} \leq 2r && \text{rapporto di approssimazione}
        \end{aligned}
    \end{equation}
    Per il punto 2, si considera un elemento $\bar{s}$ appena aggiunto a $C$.
    Nella soluzione ottima, esso si rivolge a qualche centro $C^*(\bar{s})$.
    Si considerano ora i punti che si riferiscono a quel centro: 
    \begin{equation}
        \begin{aligned}
            X &= \{ s \in S | C^*(\bar{s}) = C^*(s)\}\\
            \forall s \in X
            d(s, \bar{s}) &\leq d(s, C^*(s)) + d(C^*(s), \bar{s}) && \text{triangolare}\\
            &\leq d(s, C^*(s)) + d(C^*(\bar{s}), \bar{s})\\
            &\leq \rho^* + \rho^* = 2 \rho^* \leq 2r
        \end{aligned}
    \end{equation}
    Questo implica che dopo aver inserito $\bar{s}$ tutti i punti in $X$ sono eliminati.\\
    Visto che $|C^*| \leq K$, dopo $K$ iterazioni l'insieme $S$ sarà vuoto.  

    Il punto 3 segue dalla dimostrazione del punto 2, $a \implies b, !a \implies !b$
\end{proof}
\begin{remark}
    La dimostrazione fornisce un'idea per l'algoritmo, si potrebbe sfruttare una 
    ricerca binaria su $r$.
\end{remark}

\paragraph{Greedy Center Selection}
L'idea dell'algoritmo è prendere un punto causale, e di volta in volta 
aggiungere a $C$ il punto piu sfortunato, ovvero quello con distanza dai centri massima.

\begin{algorithm}[H]
    \SetAlgoLined
    \KwIn{$S \subseteq \Omega$, $K \in \mathbb{N}^{>0}$}
    \KwResult{Selezione dei centri}

    \If{$|S| \leq K$}{\Return{S}}
    $s = \mathit{random}(S)$\\
    $C = \{s\}$\\
    \While{$|C| < K$}{
        $\bar{s} = \max_{d(s, C )}(S)$\\
        $C.add(\bar{s})$\\
    }
    \Return{$(|C| \leq K)?\;C\;:\;\mathit{impossible}$}
     \caption{GreedyCenterSelection}
\end{algorithm}

\begin{theorem}
    Greedy Center Selection è 2 approssimante per Center Selection
\end{theorem}
\begin{proof}
    Supponiamo esista un input $(S,K)$ tale che $\rho(c) > 2\rho^*$.
    Questo implica che $\exists \hat{s} \in S$ tale che $d(\hat{s}, C) > 2\rho^*$ 

    Nelle prime $K$ iterazioni dell'algoritmo:
    \begin{equation}
        \begin{aligned}
            \bar{s}_1, \dots, \bar{s}_k && \text{centri aggiunti}\\
            \bar{C}_1, \dots, \bar{C}_k && \text{centri prima di aggiungere } s_i\\
        \end{aligned}
    \end{equation}
    Considerando la distanza del centro $s_i$:
    \begin{equation}
        \begin{aligned}
            d(\bar{s}_i,\bar{C}_i)&\geq d(\hat{s},\bar{C}_i) && \text{per massimizzazione}\\
            d(\hat{s},\bar{C}_i) &\geq d(\hat{s}, C) \geq 2\rho^* && \text{poichè espando C e per ipotesi}
        \end{aligned}
    \end{equation}
    Questo caso coincide con le prime $k$ iterazioni di Center Selection Plus con $r = \rho^*$.
    Ma quindi avrei $S \neq \emptyset$, quindi output impossibile e $r < \rho^*$
\end{proof}

\begin{theorem}
    Non esiste $\Pi \in P$ $\alpha \textit{-approssimante}$  per Center Selection con $\alpha < 2$
\end{theorem}
\begin{proof}
    Riduco a Dominating set che appartiene ad NP. Il problema 
    individua un insieme di nodi in un grafo tale che, ogni nodo
    o fa parte di tale insieme, o è vicino di un nodo che ne fa parte.

    Per tale problema:\\
    \emph{Input}: $G=(V,E)$, $K \in \mathbb{N}^{>0}$\\
    \emph{Output}: $\exists D \subseteq V $ tale che $|D| \leq K$
    tale che $\forall x \in V\setminus D, \exists y \in D, xy \in E$

    Considerando l'input di Dominating Set, i vertici costituiscono i punti 
    di Center Selection.
    Il mapping per le distanze avviene come segue:\\
    \[ 
        d(x,y) = 
        \begin{cases}
        0 & \mathit{se}\;x=y \\
        1 & \mathit{se}\;xy \in E \\
        2 & \mathit{se}\;xy \notin E
     \end{cases}
    \]
    La distanza così definita è simmetrica e rispetta la disuguaglianza triangolare:
    \begin{equation}
        \begin{aligned}
            x, y, z \in S\\
            d(x,y) \leq d(x,z) + d(z,y) && \text{le distanze sono 1,2, o 3}\\
            1,2 \leq 2,3,4  && \text{1 oppure 2 minore di 2, 3 o 4}
        \end{aligned}
    \end{equation}

    Definita la distanza, supponiamo ora di conoscere $\rho^*(S,K)$ che vale $1$ oppure $2$.
    Se $\rho^*(S,K) = 1$, allora:
            $$\exists C \subseteq S, |C| \leq K,\;t.c.,\; 
            \forall x \in S\setminus C, \exists c \in C, d(x,y) = 1$$
    Ovvero, se esiste un $C$ tale che ogni elemento al di fuori dei centri ha distanza 
    1 da uno dei punti in $C$. Se due punti $x,y$ hanno distanza 
    1, $xy \in E$.\\
    In breve, $\rho^*(S,K) = 1$ sse esiste una soluzione per Dominating Set.

    Per assurdo supponiamo esista un algoritmo $\Pi$ che $\alpha$-approssima 
    Center Selection con $\alpha < 2$.
    $$(G,K)\longrightarrow(S,K)\longrightarrow \Pi \longrightarrow \rho(S,K)$$
    La soluzione individuata da $\Pi$ sarà:
    $$\rho^*(S,K) \leq \rho(S,K) < 2\rho^*(S,K)$$
    Esistono due casi:
    \begin{itemize}
        \item $\rho^*(S,K) = 1$, allora $1\leq \rho(S,K) < 2$, ovvero esiste una soluzione 
        per Dominating Set
        \item $\rho^*(S,K) = 2$, allora $2\leq \rho(S,K) < 4$, quindi non esiste una soluzione 
        per Dominating Set
    \end{itemize}
    Questo significa che, se quell'algoritmo $\Pi$ esisitesse, riuscirei a decidere, 
    guardando il risultato $\rho(S,K)$, Dominating Set, il tutto in tempo polinomiale, assurdo.
\end{proof}