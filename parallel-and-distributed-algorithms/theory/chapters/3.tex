\textbf{Types of parallel architecture}

\begin{enumerate}
 \item \textbf{shared memory}:\\ 
  - Invisible property:\\
   We have a central clock\\
  - Communication property:\\
   Constant time communication between processors:\\
   For example: \textit{If $P_j$ wants communicate $x$ to $P_i$:}\\
   - $P_j$ writes $x$ in central memory\\
   - $P_i$ read $x$ from central memory
 \item \textbf{distributed memory}:
  - Invisible property:\\
   We have a central clock\\
  - Communication property:\\
   The communication depends on the distance between processors: if $P_j$ wants communicate with $P_i$, we should ask ourselves how many processors there are in between.
\end{enumerate}

\subsection{PRAM model (Parallel RAM)}

It is composed by:
\begin{itemize}
 \item A central memory (central clock)
 \item $n$ sequential RAMs that have thier own private memory
\end{itemize}

Types of instruction:
\begin{itemize}
 \item logical arithmetic operations
 \item operations from/to central memory:\\
 - $STORE \ R[n] \ M[n]$\\
 - $LOAD \  R[n^1] \ M[n^1]$
\end{itemize}

We are able to manipulate only the data in private memory, doing this the communication come in $O(1)$

\textbf{From of instrucitions}

\begin{algorithm}[H]
 \SetAlgoLined
 \For{$i \ \in \ I$}{$instruction_i$}
 \caption{From of instrucitions}
\end{algorithm}

The processors with index in $I$ execute $instruction_i$\\
The processors with index not in $I$ execute $null$

What's \textbf{$instruction_i$}?\\
There are two types of different architecture:
\begin{itemize}
 \item SIMD: \textit{single instruction multiple data}
 \item MIMD: \textit{multiple instruction multiple data}
\end{itemize}

$instruction_i$ is for $i \neq j$ the same instruction

Depending on the ability to access memory $M$ (central) we also have different architectures:
\begin{itemize}
 \item EREW:\\
 - read/write in the same cell of $M$
 \item CREW:\\
 - simultaneous reading
 - no simultaneous writing
 \item CRCW:\\
 - simultaneous reading/writing\\
 For simultaneous writing we have different policies:
 \begin{itemize}
  \item common: the processor can write the same data, otherwise system shutdown
  \item random: $P_i$ is chosen randomly
  \item max/min: wins $P_i$ with max/min data
  \item priority: wins $P_i$ with max priority
 \end{itemize}
\end{itemize}

\textbf{Romputing Resources}

- Sequencial: $t(n)$, $s(n)$\\
- Parallel: $p(n)$, $T(n, p(n))$

\textbf{Example of P-RAM algorithm}

\begin{algorithm}[H]
 - Assuming that array $A$ have distict value\\
 - \# \ of \ proccessors = n\\
 \SetAlgoLined
 \KwIn{$A, \ n, \ x$}
 \KwResult{$find x in A$}
 $index \gets -1$\\
 \For{$i=0; \ to \ n-1$} {
  $P_i$: \If{$A[i] = x$} {
   $index = i$
  } 
 }
 \Return{$index$}
 \caption{Find}
\end{algorithm}

Parallel time = constant

If $A$ repeating elements $\rightarrow$ CRCW

\textbf{Informal definition}

$P(n)$ = number of processors required on input of length $n$.

$T(n, P(n))$ = Time required by an input of length $n$ and $P(n)$ processors. (Worst case)

\textbf{Formal definition}

$t_i(n) = max\{(t_i)^{(j)}_{(n)} \ | \ 1 \leq j \leq P(n)\}$

$T(n, P(n)) = \sum_{i=1}^{k(n)}{t_i(n)}$

$T$ depends on $k(n)$

$T$ depends on the input length $[logarithmic / uniform \ cost)$

$T$ depends on $P(n)$