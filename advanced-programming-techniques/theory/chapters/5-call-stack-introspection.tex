\section{Call Stack Introspection}

\subsection{Call Stack Introspection}

\subsubsection{State Introspection}

Introspection can't be only applied to the application structure.

Data about the program execution can be introspected as well:
\begin{itemize}
	\item the execution state; and
	\item the call stack
\end{itemize}

Each thread has a call stack consisting of stack frames

Call stack introspection allows a thread to examine its context
\begin{itemize}
	\item the execution trace, and the current frame
\end{itemize}

\subsubsection{Call Stack Reification: Throwable \& StackTraceElement}

In Java there is no accessibile call stack meta-object.

So, where is our entry point?
\begin{itemize}
	\item when an instance of \textbf{Throwable} is created, the call stack as an array of \textbf{StackTraceElement}
\end{itemize}

By writing:
\begin{lstlisting}[language=Java]
new Throwable().getStackTrace()
\end{lstlisting}

We have access to a representation of the call stack when the Throwable was created.

The getStackTrace() method returns the current call stack as an array of StackTraceElement. The first is the current frame.

\subsubsection{Call Stack Reification: Throwable \& StackTraceElement (Cont'd)}
From a frame we can get:
\begin{itemize}
	\item the file name containing the execution point (getFileName())
	\item the line number where the call occurs (getLineNumber())
	\item the name of the class and of the method containing the execution point (getClassName() and getMethodName())
\end{itemize}

\begin{lstlisting}[language=Java]
public class ABC {
public void a() {b();}
public void b() {c();}
private void c() {
	for(StackTraceElement f: new Throwable().getStackTrace()) System.out.println(f);
}
public static void main(String args[]) { new ABC().a(); }
}
\end{lstlisting}

\begin{lstlisting}[language=Java]
[14:16]cazzola@hymir:~/tsp>java ABC
ABC.c(ABC.java:5)
ABC.b(ABC.java:3)
ABC.a(ABC.java:2)
ABC.main(ABC.java:7)
\end{lstlisting}

\subsubsection{The Logging Facility (Naive Version)}
We want to add the logging facility to all our applications

The interface to the logging facility is:

\begin{lstlisting}[language=Java]
public interface Logger {
	void logRecord(String clazz, String method, int lineno, String msg, int type);
	void logProblem(String clazz, String method, int lineno, Throwable prob);
}
\end{lstlisting}

The logging facility has to be called at each critical point

\begin{lstlisting}[language=Java]
public class Account {
private Logger log = new LoggerImpl();
			...
public void withdrawal (int sum) {
			...
		this.log.logRecord("Account", "withdrawal", 23, "Execution ...", 0);
	}
}
\end{lstlisting}

This approach is:
\begin{itemize}
	\item boring
	\item fragile, e.g., the line number easily changes, and
	\item error-prone
\end{itemize}

\subsubsection{The Logging Facility (Version with Call Stack Introspection)}

Inspecting the call stack helps in avoiding confusion
\begin{itemize}
	\item class, method and line number can be got inspecting the frame
\end{itemize}

\begin{lstlisting}[language=Java]
public interface Logger {
	void logRecord(String msg, int type);
	void logProblem(Throwable prob);
}
\end{lstlisting}

That is impleented as:
\begin{lstlisting}[language=Java]
public class LoggerImpl implements Logger {
	public void logRecord(String message, int logRecordType) {
		StackTraceElement f = new Throwable().getStackTrace()[1];
		String callerClassName = f.getClassName();
		String callerMethodName = f.getMethodName();
		int callerLineNumber = f.getLineNumber();
		// write of log record goes here.
	}
}
\end{lstlisting}

\subsubsection{The Invariant Checking Facility: Problem Definition}

Let’s define a class VisiblePoint
\begin{itemize}
	\item the instances are legit when their coordinates are within the display limits
	\item the limits and the way to check them are defined by the Visible interface
\end{itemize}

\begin{lstlisting}[language=Java]
public interface Visible {
	public final int XMIN = -1080 ;
	public final int XMAX = 1080 ;
	public final int YMIN = -1920 ;
	public final int YMAX = 1920 ;
	default boolean isvisiblex(int x) { return (x>= XMIN && x <= XMAX) ; }
	default boolean isvisibley(int y) { return (y>= YMIN && y <= YMAX) ; }
}
\end{lstlisting}

\begin{lstlisting}[language=Java]
public class VisiblePoint implements Visible {
	private int x, y;
	public VisiblePoint(int x, int y) {
		this.x = x; this.y=y;
		assert isvisiblex(this.x) && isvisibley(this.y):
			"x or y coordinates outside the display margins" ;
	}
	public int getX() { return this.x; }
	public int getY() { return this.y; }
	public void setX(int x) { this.x=x; }
	public void setY(int y) { this.y=y; }
}
\end{lstlisting}

\subsubsection{The Invariant Checking Facility: Invariant Definition}
An invariant is a property that must hold for the whole instance' lifecycle.
- e.g., for a VisiblePoint must always hold that its coordinates are within the display borders

The class to be checked must implement the interface:

\begin{lstlisting}[language=Java]
public interface InvariantSupporter { boolean invariant(); }
\end{lstlisting}

The invariant() must be invoked at the begin/end of each method

\begin{lstlisting}[language=Java]
public class VisiblePoint implements Visible, InvariantSupporter {
		...
	public boolean invariant() { return isvisiblex(getX()) && isvisibley(getY()) ; }
	public int getX() {
		InvariantChecker.checkInvariant(this); int result = this.x;
		InvariantChecker.checkInvariant(this);
		return result ;
	}
		...
	public void setY(int y) {
			InvariantChecker.checkInvariant(this); this.y=y;
		InvariantChecker.checkInvariant(this);
	}
}
\end{lstlisting}

\subsubsection{The Invariant Checking Facility: InvariantChecker (Naive)}

The check is carried out by a class InvariantChecker

\begin{lstlisting}[language=Java]
public class InvariantChecker {
	public static void checkInvariant(InvariantSupporter obj) {
		if (!obj.invariant()) throw new IllegalStateException("invariant failure");
	}
}
\end{lstlisting}

\begin{lstlisting}[language=Java]
public class MainInvariantChecker {
	public static void main(String[] args) {
		VisiblePoint p = new VisiblePoint(-7, 25);
		System.out.println("Point p is: ("+p.getX()+", "+p.getY()+")");
		p.setX(-20); System.out.println("New Point is: ("+p.getX()+", "+p.getY()+")");
		p.setX(-2000); System.out.println("New Point is: ("+p.getX()+", "+p.getY()+")");
	}
}
\end{lstlisting}

\begin{lstlisting}[language=Java]
[11:16]cazzola@hymir:~/tsp/v2>java MainInvariantChecker
Exception in thread "main" java.lang.StackOverflowError
	at InvariantChecker.checkInvariant(InvariantChecker.java:3)
	at VisiblePoint.getX(VisiblePoint.java:12)
	at VisiblePoint.invariant(VisiblePoint.java:9)
		...
\end{lstlisting}

Unfortunately,
\begin{itemize}
	\item invariant() uses a method of VisiblePoint that is checked as well
	\item this creates an infinite loop of invariant checkings
\end{itemize}

\subsubsection{The Invariant Checking Facility: InvariantChecker (with CSI)}

Problem
\begin{itemize}
	\item there is a loop when the invariant() invokes a method under invariant check
\end{itemize}

Solution
\begin{itemize}
	\item inspecting the call stack before invoking the invariant() method looking for a loop
\end{itemize}

\begin{lstlisting}[language=Java]
public class InvariantChecker {
	public static void checkInvariant(InvariantSupporter obj) {
		StackTraceElement[] ste = (new Throwable()).getStackTrace();
		for (int i = 1 ; i<ste.length; i++)
			if (ste[i].getClassName().equals("InvariantChecker") &&
				ste[i].getMethodName().equals("checkInvariant") ) return ;
		if ( !obj.invariant() )
			throw new IllegalStateException("invariant failure");
	}
}
\end{lstlisting}

\begin{lstlisting}[language=Java]
[11:15]cazzola@hymir:~/tsp/v3>java MainInvariantChecker
Point p is: (-7, 25)
New Point is: (-20, 25)
Exception in thread "main" java.lang.IllegalStateException: invariant failure
	at InvariantChecker.checkInvariant(InvariantChecker.java:9)
	at VisiblePoint.setX(VisiblePoint.java:27)
	at MainInvariantChecker.main(MainInvariantChecker.java:7)
\end{lstlisting}

\subsubsection{Selective Accessibility Permission Granting}

Problem
\begin{itemize}
	\item accessibility permissions are all allowed or negated
\end{itemize}

Solution
\begin{itemize}
	\item call stack inspection when the permissions should be enabled
\end{itemize}

\begin{lstlisting}[language=Java]
public class SelectiveAccessibilityCheck {
	public static void main(String[] args) throws Exception {
		System.setSecurityManager(new SecurityManager() {
			public void checkPermission(Permission p) {
				if (p instanceof ReflectPermission && "suppressAccessChecks".equals(p.getName()))
				for (StackTraceElement e : Thread.currentThread().getStackTrace())
					if ("SelectiveAccessibilityCheck".equals(e.getClassName()) &&
"setName".equals(e.getMethodName())) throw new SecurityException();
	}
});
	Employee eleonor = new Employee("Eleonor", "Runedottir"); System.out.println(eleonor);
	setSurname(eleonor, "Odindottir"); System.out.println(eleonor);
	setName(eleonor, "Angela"); System.out.println(eleonor);
}

private static void setName(Employee e, String n) throws Exception {
	Field name = Employee.class.getDeclaredField("name") ;
	name.setAccessible(true) ; name.set(e, n); }
private static void setSurname(Employee e, String s) throws Exception {
	Field surname = Employee.class.getDeclaredField("surname") ;
	surname.setAccessible(true) ; surname.set(e, s); }
}
\end{lstlisting}

\begin{lstlisting}[language=Java]
[12:54]cazzola@hymir:~/tsp>java SelectiveAccessibilityCheck
Employee: Eleonor Runedottir
Employee: Eleonor Odindottir
\end{lstlisting}