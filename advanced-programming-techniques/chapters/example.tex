\section{section}
...

\subsection{subsection}
	
\paragraph{paragraph}

\begin{remark}
    ...
\end{remark}

\subsubsection{subsubsection}

\subsubsection{subsubsection}

\paragraph{paragraph}

\subsection{subsection}
...
\begin{enumerate}
    \item 
    \item
    \item 
\end{enumerate}

\begin{itemize}
	\item 
    \item 
\end{itemize}

\begin{theorem}
    $PO \subseteq NPO$
\end{theorem}
\begin{proof}
    ...
\end{proof}

\begin{equation}
    \begin{aligned}
        \mathit{APX} = \{\Pi | \Pi \mathit{\;di\;ottimizzazione\;t.c.\;}
\exists \rho \geq 1, A, \\\mathit{t.c\;} x\rightarrow A \rightarrow y(x)\;\mathit{con}\;R_\Pi(x, y) \leq \rho\}
    \end{aligned}
\end{equation}


\paragraph{Algoritmo di risoluzione}
L'algoritmo di risoluzione è abbastanza semplice e si basa sulla 
ricerca di un cammino aumentante.

\begin{algorithm}[H]
    \SetAlgoLined
    \KwIn{$G=(V,E)$}
    \KwResult{Matching $M$ per $G$}
     $M \gets \emptyset$\\
     \While{$\Pi = \mathit{findAugmenting(G)}$}{
        $M.update(\Pi)$
     }
     \Return{M}
     \caption{BiMaxMatching}
\end{algorithm}


\subsection{Tecniche greedy}
Nella sezione a seguire si presentano problemi di ottimizzazione per cui 
tecniche greedy funzionano abbastanza bene. 
I problemi affrontati sono quelli di \emph{Load Balancing}, \emph{Center Selection} e \emph{Set Cover}.

\subsubsection{Load balancing}
\label{lb}
Il problema di Load Balancing può essere visto come il compito
di assegnare a macchine dei lavori da compiere, che richiedono del tempo, 
in modo da minimizzare il tempo totale.


\begin{theorem}
    Load Balancing è NPO completo
\end{theorem}
Bisogna perciò trovare un modo di approssimare una soluzione.
\paragraph{Greedy balance}
Il primo approccio alla risoluzione del problema è quello di assegnare la prossima
task alla macchina più scarica in questo momento.

\begin{algorithm}[H]
    \SetAlgoLined
    \KwIn{$M$ numero di macchine, $t_0, \dots, t_n$ task}
    \KwResult{Assegnamento delle task alle macchine}
     $L_i \gets 0\;\forall i \in M$\\
     $\alpha \gets \emptyset$\\
     \For{$j = 0, \dots, n$}{
         $\hat{i} = \min(L_i)$\\
         $\alpha(j) = \hat{i}$\\
         $L_{\hat{i}} \pluseq t_j$
     }
     \Return{$\alpha$}
     \caption{GreedyBalance}
\end{algorithm}

\begin{lstlisting} %[language=Python, caption=Python example]

import numpy as np
    
def incmatrix(genl1,genl2):
    m = len(genl1)
    n = len(genl2)
    M = None #to become the incidence matrix
    VT = np.zeros((n*m,1), int)  #dummy variable
    
    #compute the bitwise xor matrix
    M1 = bitxormatrix(genl1)
    M2 = np.triu(bitxormatrix(genl2),1) 

    for i in range(m-1):
        for j in range(i+1, m):
            [r,c] = np.where(M2 == M1[i,j])
            for k in range(len(r)):
                VT[(i)*n + r[k]] = 1;
                VT[(i)*n + c[k]] = 1;
                VT[(j)*n + r[k]] = 1;
                VT[(j)*n + c[k]] = 1;
                
                if M is None:
                    M = np.copy(VT)
                else:
                    M = np.concatenate((M, VT), 1)
                
                VT = np.zeros((n*m,1), int)
    
    return M
\end{lstlisting}

%%%%%%%%%%%%%%%%%
\usepackage{listings}
\usepackage{xcolor}

\definecolor{codegreen}{rgb}{0,0.6,0}
\definecolor{codegray}{rgb}{0.5,0.5,0.5}
\definecolor{codepurple}{rgb}{0.58,0,0.82}
\definecolor{backcolour}{rgb}{0.95,0.95,0.92}

\lstdefinestyle{mystyle}{
    backgroundcolor=\color{backcolour},   
    commentstyle=\color{codegreen},
    keywordstyle=\color{magenta},
    numberstyle=\tiny\color{codegray},
    stringstyle=\color{codepurple},
    basicstyle=\ttfamily\footnotesize,
    breakatwhitespace=false,         
    breaklines=true,                 
    captionpos=b,                    
    keepspaces=true,                 
    numbers=left,                    
    numbersep=5pt,                  
    showspaces=false,                
    showstringspaces=false,
    showtabs=false,                  
    tabsize=2
}

\lstset{style=mystyle}