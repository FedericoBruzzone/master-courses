\section{Object Oriented Programming in Python}

Classes, Inheritance & Polymorphism

\subsection{Object-Oriented Programming}

\subsubsection{Introduction}

Python is a multi-paradigm programmin language

Many claims that:

\begin{center}
Python is object-oriented
\end{center}

Python is just \textbf{object-based} but we can use it as if it is object-oriented

Look at

\begin{center}
\textbf{Reference}
Peter Wagner
\textbf{Dimensions of Object-Based Language Design}
In Proceegings of OOPSLA'87, pp. 168-182, October 1987.
\end{center}

for the differences

\subsubsection{Wagner’s OO Taxonomy: Objects, Classes and Inheritance}

\textbf{Objects}
An object has a set of operations and a state that remembers the effect of the operations

\textbf{Class}
A class is a template from which objects may be created

\begin{itemize}
	\item object of the same class have common operations and (therefore) uniform behavior
	\item Class expose a set of operations (public interface) to its clients
\end{itemize}

\textbf{Inheritance}
A class may inherit operations from superclasses and its operations inherited by subclasses

\begin{itemize}
	\item  inheritance can be single or multiple
\end{itemize}

\subsubsection{Wagner’s OO Taxonomy (Cont.’d)}

Wagner suggests 3 classes for programming languages:

\begin{itemize}
	\item object-based = object
	\item class-based = object + classes
	\item object-oriented = onject + classes + inheritance
\end{itemize}

\textbf{Data Abstraction}
A data abstraction is an object whose state is accessible only through its operations

\begin{itemize}
	\iem this concept brings forth to the data hifing property
\end{itemize}

\textbf{Delegation}
Delegation is a mechanism to delegate responsibility for performing an operation to one or more designed ancestors

\begin{itemize}
		\item note that ancestors are not always designed by inheritance in this case it is called clientship
\end{itemize}

\subsubsection{Class Definition: Rectangle}

\begin{lstlisting}[language=Python]
	class rectangle:
		def __init__(self, width, height):
			self._width=width
			self._height=height
		def calculate_area(self):
			return self._width*self._height
		def calculate_perimeter(self):
			return 2*(self._height+self._width)
		def __str__(self):
			return "I’m a Rectangle! My sides are: {0}, {1}\nMy area is {2}".\
			format(self._width,self._height, self.calculate_area())
\end{lstlisting}

\begin{lstlisting}[language=Python]
>>> from rectangle import rectangle
>>> r = rectangle(7,42)
>>> print(r)
I’m a Rectangle! My sides are: 7, 42
My area is 294
\end{lstlisting}

\subsubsection{Inheritance}

Inheritance 


