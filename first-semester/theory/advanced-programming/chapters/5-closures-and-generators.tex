\section{Closures and generators}

\subsection{Closures}

\subsubsection{On a real problem}

\textbf{English, from singular to plural}
\begin{itemize}
	\item if a word ends in S, X or Z, add ES, e.g., fax becomes faxes;
	\item if a word ends in a noisy H,add ES,e.g.,coachbecomescoaches;
	\item if it ends in a silent H, just add S, e.g., cheetah becomes cheetahs
	\item if a word ends in Y thath sound like I, change the T to IES, e.g., vacancy becomes vacancies;
	\item if the Y is combined with a vowel to sound like something else,just  add S,e.g.,day becomes days;
	\item if all else fails, just add S and hope for the best;
\end{itemize}

\textbf{We will design a Python module that automatically pluralizes English nouns}

\begin{itemize}
	\item All these characteristics make for more rapidly developed, shorter, and less bug-prone code.
	\item A lot easier to prove formal properties of functional languages and programs than of imperative languages and programs.
\end{itemize}

\subsubsection{Regular Expression}
A \textbf{regular expression} is a pattern to describe strings
\begin{itemize}
	\item the functions in the \textbf{re} module enables us to check if a regular expression matches a string and to return the result of the match
\end{itemize}

Few bytes of syntax:
\begin{itemize}
	\item `.` any character but a newline
	\item `\^` the begin of the string
	\item `\$` the end of the string
	\item `*`, `+` 0 (or 1) or more repetitions of the preceding RE
	\item `?` 0 or 1 repetitions of the preceding REs
	\item [] a set of characters
	\item {} matching group
\end{itemize}

\textbf{RE at work}
\begin{lstlisting}[language=Python]
>>> email = `cazzola@dremove_thisi.unimi.it`
>>> import re
>>> m = re.search("remove_this", email)
>>> email[:m.start()]+email[m.end():]
`cazzola@di.unimi.it`
\end{lstlisting}

\subsubsection{Do Some Abstraction: A List of Functions}

To abstract we have
\begin{itemize}
	\item to limit the number of tests to be done
	\item to generalize the approach
\end{itemize}

\begin{lstlisting}[language=Python]
import re

def match_sxz(noun): return re.search('[sxz]$', noun)
def apply_sxz(noun): return re.sub('$', 'es', noun)
def match_h(noun): return re.search('[^aeioudgkprt]h$', noun)
def apply_h(noun): return re.sub('$', 'es', noun)
def match_y(noun): return re.search('[^aeiou]y$', noun)
def apply_y(noun): return re.sub('y$', 'ies', noun)
def match_default(noun): return True
def apply_default(noun): return noun + 's'

rules = ((match_sxz, apply_sxz), (match_h, apply_h), (match_y, apply_y), (match_default, apply_default))

def plural(noun):
	for matches_rule, apply_rule in rules:
		if matches_rule(noun):
			return apply_rule(noun)
\end{lstlisting}

Advantage:

- to add new rules simply means to add a couple of function and a tuple in the rules tuple

\subsubsection{Do Some Abstraction: A List of Patterns}
To do better, we have

- to avoid to write the single functions (boring \& error-prone task)

\begin{lstlisting}[language=Python]
import re

def build_match_and_apply_functions(pattern, search, replace):
	def matches_rule(word):
		return re.search(pattern, word)
	apply_rule = lambda word : \
		re.sub(search, replace, word)
	return (matches_rule, apply_rule)

patterns = ( \
	('[sxz]$', '$', 'es'), ('[^aeioudgkprt]h$', '$', 'es'),
	('(qu|[^aeiou])y$', 'y$', 'ies'), ('$', '$', 's')
)

rules = [ \
	build_match_and_apply_functions(pattern, search, replace)
		for (pattern, search, replace) in patterns ]
\end{lstlisting}

\textbf{The technique of binding a value within the scope definition to a value in the outside scope is named closures}

\begin{itemize}
	\item It fixes the value of some variables in the body of the functions it builds:
	- both matches \_ rule and apply \_ rule take one parameter (word), they act on that plus three other values (pattern, search and replace) which were set when the functions are built
\end{itemize}

\subsubsection{Do Some Abstraction: A File of Patterns}

\textbf{Separate data from code}

- By moving the patterns in a separate file

\begin{lstlisting}[language=Python]
[15:59]cazzola@hymir:~/esercizi-pa>cat plural-rules.txt
[sxz]$           $ es
[^aeioudgkprt]h$ $ es
[^aeiou]y$      y$ ies
$                $ s
\end{lstlisting}

\textbf{Everything is still the same but}

- how is the rules list filled?

\begin{lstlisting}[language=Python]
rules = []
with open('plural-rules.txt', encoding='utf-8') as pattern_file:
	for line in pattern_file:
		pattern, search, replace = line.split(None, 3)
		rules.append(build_match_and_apply_functions(pattern, search, replace))
\end{lstlisting}

\textbf{Benefits \& Drawbacks}

\begin{itemize}
	\item no need to change the code in order to add a new rule
	\item to read a file is slower than to hardwire the data in the code
\end{itemize}

\subsection{Generators}

\subsubsection{Introduction by Example}
 A \textbf{generator} is a function that generates a value at a time
 
 - a sort of resumable function or function with a memory
 
\begin{lstlisting}[language=Python]
def make_counter(x):
	print('entering make_counter')
	while True:
		yield x
		print('incrementing x’ù')
		x = x + 1
\end{lstlisting}

Let look at what happens here
\begin{lstlisting}[language=Python]
>>> import counter
>>> counter = counter.make_counter(2)
>>> next(counter)
entering make_counter
2
>>> next(counter)
incrementing x
3
\end{lstlisting}

\begin{itemize}
	\item a call to the function initializes the generator
	\item the next() will “synchronize” with the yield statement
	- the yield suspends the function execution and returns a value
	- the next() resumes the computation from the yield and continues until it reaches another yield or the function end
\end{itemize}

\subsubsection{Finbonacci's generator}
\begin{lstlisting}[language=Python]
def gfib(max):
	a, b = 0, 1
	while a < max:
		yield a
		a, b = b, a + b
		
if __name__ == "__main__":
	for n in gfib(1000):
		print(n, end=’ ’)
	print()
\end{lstlisting}

\begin{lstlisting}[language=Python]
[15:43]cazzola@hymir:~/esercizi-pa>python3 gfib.py
0 1 1 2 3 5 8 13 21 34 55 89 144 233 377 610 987

>>> import gfib
>>> list(gfib.gfib(1000))
[0, 1, 1, 2, 3, 5, 8, 13, 21, 34, 55, 89, 144, 233, 377, 610, 987]
\end{lstlisting}

\begin{itemize}
	\item a generator can be used in a for statement, the next() is automatically called at each iteration
	\item the list constructor has a similar behavior
\end{itemize}

\subsubsection{Pluralizes Via Generators}

\begin{lstlisting}[language=Python]
def rules(rules_filename):
	with open(rules_filename, encoding=’utf-8’) as pattern_file:
		for line in pattern_file:
			pattern, search, replace = line.split(None, 3)
			yield build_match_and_apply_functions(pattern, search, replace)
def plural(noun, rules_filename=’plural-rules.txt’):
	for matches_rule, apply_rule in rules(rules_filename):
			if matches_rule(noun):
				return apply_rule(noun)
	raise ValueError(’no matching rule for {0}’.format(noun))
\end{lstlisting}

\textbf{Benefits \& Drawbacks}
\begin{itemize}
	\item shorter start-up time (it just reads a row not the whole file)
	\item performance losses (every call to plural() reopensthe file and reads it from the beginning again)
\end{itemize}

\textbf{To get the benefits from both approaches you need to define your own iterator}